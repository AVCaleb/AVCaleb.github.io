% !TEX program = xelatex
\documentclass[12pt, a4paper]{book}

% Geometry
\usepackage[margin=2.5cm]{geometry}

% ============================================
% IMPORTANT: Load these packages BEFORE polyglossia/bidi
% The bidi package (loaded by polyglossia for Hebrew) requires
% xcolor and graphicx to be loaded first.
% It also requires hyperref, fancyhdr, pgf, tikz, titlesec to be loaded BEFORE it.
% ============================================
\usepackage{graphicx}
\usepackage{xcolor}
\definecolor{accentcolor}{RGB}{52, 152, 219}
\definecolor{lightgray}{RGB}{245, 245, 245}

% Unicode and fonts
\usepackage{fontspec}
\usepackage{xeCJK}
\usepackage{zhnumber}

% Set fonts
\setmainfont{GaramondMT-W04}
\setsansfont{Helvetica Neue}
\setmonofont{Palatino}
\setCJKmainfont{STSong}
\setCJKsansfont{STHeiti}
\setCJKmonofont{KaiTi}

% Typography
\usepackage{indentfirst}
\usepackage{setspace}
\setstretch{1.3}
\xeCJKsetup{CJKglue={\hskip 0.1em plus 0.05em minus 0.05em}} % 调整中文字符间距
% \setlength{\parskip}{0.8em}
\AtBeginDocument{\settowidth{\parindent}{两个}}
\raggedbottom
% Headers and footers
\usepackage{fancyhdr}
\pagestyle{fancy}
\fancyhf{}
\fancyhead[LE,RO]{\footnotesize 史百克}
\fancyhead[RE]{\footnotesize 约翰福音注}
\fancyhead[LO]{\footnotesize \leftmark}
\fancyfoot[C]{\footnotesize\thepage}
\renewcommand{\headrulewidth}{0.4pt}
\renewcommand{\chaptermark}[1]{\markboth{第\zhnum{chapter}章\quad #1}{}}

% Chapter styling
\usepackage{titlesec}

% Boxes for scripture
\usepackage{tcolorbox}
\tcbuselibrary{skins,breakable}

% For parallel text
\usepackage{paracol}

% Hyperlinks
\usepackage[bookmarks=true]{hyperref}
\hypersetup{
    colorlinks=true,
    linkcolor=black,
    urlcolor=accentcolor,
    pdfauthor={T. Austin Sparks},
    pdftitle={Gospel of John}
}

% Greek and Hebrew support (loads bidi package internally)
% MUST BE LOADED AFTER hyperref, fancyhdr, etc.
\usepackage{polyglossia}
\setmainlanguage{english}
\setotherlanguage{greek}
\setotherlanguage{hebrew}
\newfontfamily\greekfont{Times New Roman}
\newfontfamily\hebrewfont{Times New Roman}

% Chapter styling (Applied here to override potential polyglossia resets)
\titleformat{\chapter}[display]
  {\normalfont\huge\bfseries\centering}
  {第\zhnum{chapter}章}{20pt}{\Huge}
\titlespacing*{\chapter}{0pt}{50pt}{40pt}

% Table of contents depth
\setcounter{tocdepth}{2}
\AtBeginDocument{\addto\captionsenglish{\renewcommand{\contentsname}{目录}}}

% Custom commands
\newcommand{\scripture}[1]{%
    \begin{tcolorbox}[
        enhanced,
        colback=accentcolor!8,
        colframe=white,
        boxrule=0.5pt,
        sharp corners,
        left=10pt,
        right=0pt,
        top=10pt,
        bottom=10pt,
        boxsep=0pt,
        breakable
    ]
    \ttfamily #1
    \end{tcolorbox}
}

\newcommand{\outlinesection}[1]{
    \begin{center}
        \textbf{#1}
    \end{center}
}

% Custom outline environment
\newenvironment{outline}{%
    \par
    \ttfamily
    \setlength{\parindent}{0pt}%
    \obeylines
}{%
    \par
}

% Footnotes
\usepackage[bottom]{footmisc}
\renewcommand{\footnoterule}{\vfill\kern-3pt\hrule width 0.4\columnwidth\kern2.6pt}

\begin{document}

\frontmatter

\begin{titlepage}
    \centering
    \vspace*{3cm}

    {\Huge\bfseries 约翰福音注}

    \vspace{2cm}

    {\large 史百克}

    \vfill

    {\large \ttfamily 1900年版}

\end{titlepage}

\tableofcontents
\newpage

\mainmatter

\chapter{教会的使命}

\scripture{\textbf{约翰福音 13:1-38} 1 逾越节以前,耶稣知道自己离世归父的时候到了。他既然爱世间属自己的人,就爱他们到底。2 吃晚饭的时候(魔鬼已将卖耶稣的意思放在西门的儿子加略人犹大心里),3 耶稣知道父已将万有交在他手里,且知道自己是从神出来的,又要归到神那里去,4 就离席站起来脱了衣服,拿一条手巾束腰。5 随后把水倒在盆里,就洗门徒的脚,并用自己所束的手巾擦干。6 挨到西门彼得,彼得对他说:「主啊,你洗我的脚吗?」7 耶稣回答说:「我所做的,你如今不知道,后来必明白。」8 彼得说:「你永不可洗我的脚!」耶稣说:「我若不洗你,你就与我无分了。」9 西门彼得说:「主啊,不但我的脚,连手和头也要洗!」10 耶稣说:「凡洗过澡的人,只要把脚一洗,全身就干净了;你们是干净的,然而不都是干净的。」11 耶稣原知道要卖他的是谁,所以说:「你们不都是干净的。」12 耶稣洗完了他们的脚,就穿上衣服,又坐下,对他们说:「我向你们所做的,你们明白吗?13 你们称呼我夫子,称呼我主,你们说的不错,我本来是。14 我是你们的主,你们的夫子,尚且洗你们的脚,你们也当彼此洗脚。15 我给你们作了榜样,叫你们照着我向你们所做的去做。16 我实实在在地告诉你们:仆人不能大于主人,差人也不能大于差他的人。17 你们既知道这事,若是去行就有福了。18 我这话不是指着你们众人说的,我知道我所拣选的是谁。现在要应验经上的话,说:『同我吃饭的人用脚踢我。』19 如今事情还没有发生,我要先告诉你们,叫你们到事情发生的时候可以信我是基督。20 我实实在在地告诉你们:有人接待我所差遣的,就是接待我;接待我,就是接待那差遣我的。」21 耶稣说了这话,心里忧愁,就明说:「我实实在在地告诉你们:你们中间有一个人要卖我了。」22 门徒彼此对看,猜不透所说的是谁。23 有一个门徒,是耶稣所爱的,侧身挨近耶稣的怀里。24 西门彼得点头对他说:「你告诉我们,主是指着谁说的。」25 那门徒便就势靠着耶稣的胸膛,问他说:「主啊,是谁呢?」26 耶稣回答说:「我蘸一点饼给谁,就是谁。」耶稣就蘸了一点饼,递给加略人西门的儿子犹大。27 他吃了以后,撒旦就入了他的心。耶稣便对他说:「你所做的,快做吧!」28 同席的人没有一个知道是为什么对他说这话。29 有人因犹大带着钱囊,以为耶稣是对他说:「你去买我们过节所应用的东西」,或是叫他拿什么周济穷人。30 犹大受了那点饼,立刻就出去。那时候是夜间了。31 他既出去,耶稣就说:「如今人子得了荣耀,神在人子身上也得了荣耀。32 神要因自己荣耀人子,并且要快快地荣耀他。33 小子们,我还有不多的时候与你们同在;后来你们要找我,但我所去的地方你们不能到。这话我曾对犹太人说过,如今也照样对你们说。34 我赐给你们一条新命令,乃是叫你们彼此相爱;我怎样爱你们,你们也要怎样相爱。35 你们若有彼此相爱的心,众人因此就认出你们是我的门徒了。」36 西门彼得问耶稣说:「主往哪里去?」耶稣回答说:「我所去的地方,你现在不能跟我去,后来却要跟我去。」37 彼得说:「主啊,我为什么现在不能跟你去?我愿意为你舍命!」38 耶稣说:「你愿意为我舍命吗?我实实在在地告诉你:鸡叫以先,你要三次不认我。}

我们必须仔细查看几件事,这将有助于我们深入领会本章及下一章的神圣含意与内容。这两章构成了福音书中一个确定的段落。

让我们先划出本章中的几件事。首先,要注意关于主即将离开世界的题及:第1节,「祂的时候到了,祂要离开这世界……」;第3节,「……知道父已将万有交在祂手里,且祂自己……要归到神那里」;第33节,「小子们,我还有不多的时候与你们同在……」;第36节,「西门彼得问祂说,主啊,你往哪里去?耶稣回答说,我所去的地方……」。

若回想第12章第35节,你会记得祂曾说过颇为相似的话:「耶稣对他们说:『光在你们中间还有不多的时候……』」

第二件要看见的事是,现在不能跟随祂,以及对于「后来」的题及:第7节,「耶稣回答说:『我所做的,你如今不知道,后来必明白』」;第36节,「我所去的地方……你后来却要跟我去。」

第三件要注意的事是本章开头的内容:洗脚。请记住这是本章一个确定的特征;主洗了门徒的脚,然后命令他们也当彼此照样做。

还有一件事要注意,就是新命令,以及与之相关向着世人的见证:第34-35节,「我赐给你们一条新命令,乃是叫你们彼此相爱;我怎样爱你们,你们也要怎样相爱。你们若有彼此相爱的心,众人因此就认出你们是我的门徒了。」

接着在三节经文中我们当注意另一件事:那就是与基督同分并服事。第8节,「我若不洗你,你就与我无分了」——我认为最后这三个字支配着本段的全部内容:「与我有分」。第13-14节,「你们称呼我夫子,称呼我主:你们说的不错;我本来是。我是你们的主,你们的夫子……」;第16节,「仆人不能大于主人……」。

记下这些要点后,让我们来看本章的一般信息。就教会而言,它标志着比第12章更进了一步。

\section{教会}

让我们再次定义教会。根据约翰福音12章,教会是那些藉着主耶稣的死与复活而与祂有生命交通的一群人。这就是教会,就这一点而言,本章带我们更向前进。在第12章中,是与主耶稣复活的联合。在第13和14章中,所引入的是与主耶稣升天的联合。这就是为什么我们注意到第13章中的四节经文和第12章中的一节经文都题及祂即将离开。这是一个支配性的主题,在此处备受关注。

第14章很大程度上都在讲论主的离去、父的家(「我去预备地方」),以及本章其余部分都关乎主到父那里去、离开这个世界。主进入天上是此处备受关注的主题,并且支配着这里的一切。

这一切都是为了表明:教会,即信徒藉着死与复活与基督的交通,乃是一件属天的事;既在本段中提到服事,教会的服事就与天上的基督有关。主换句话是在说,祂和你们都有一个新的地位,这个新地位是在世界之外,是在天上。祂说:「我要去天上,我要到父那里去,我要离开世界,你们在灵里要与我一同出去,你们在灵里要去与我在一个新的地位上联合。所以,从此以后你们的一切——因为我在那个地方作你们的元首、你们的主、你们的夫子——都将是属天的,是出于天的。」

若回想第12章第31节,你会记得这些话:「……现在这世界受审判了」。因此,这个世界处于审判之下,「我要离开世界,离开审判的范围,离开定罪的领域,我要到父那里去,而你们,我的教会,将在灵里与我联合,被带出审判的范围,离开世界。」

如果有一件事对教会来说是真实的,有一件事在新约后续启示中被清楚阐明(在那里这些事的教义得到发展),那就是教会是在审判和定罪的范围之外。主的百姓已经被迁出黑暗的权势,黑暗的国度,进入神爱子的国。祂在那里,而我们被描述为在基督里在天上,而在基督耶稣里就不再有定罪了。「全世界都卧在那恶者手下」,因此全世界都处于审判之下。但我们不属这世界,我们在基督里,被带出来了。教会是一件属天的事,已经脱离了世界的审判。

现在,这已成为主百姓的责任。除非我们在灵里处于那个地位,即在世界之外,否则就不可能讨神的喜悦。凡与世界为友的(即,在任何方面与世界有心系之处),就是与神为敌。因此,除非我们在灵里真正脱离那称为世界的事物,否则我们就无法知道神的喜悦。换句话说,我们必须现在就以天上为我们的家、我们的本乡、我们的安息之所、我们一切供应的源头,才能享受主的喜悦;当我们以这种方式认识天上时,我们就知道父的喜悦。

\section{寻求上面的事}

由此产生另一件事;就在我们的生命是属天的,并且我们应验主在歌罗西书3:1-2中的话语的度量上:「所以你们若真与基督一同复活,就当求在上面的事,那里有基督坐在神的右边。你们要思念上面的事,不要思念地上的事……」就在我们的生命是属天的生命,并且我们活在与基督属天的联合中,从祂那里汲取一切资源的度量上,我们就会明白属天的事。换句话说,在灵里自愿接触、交通、活在这个世界是一种使属灵知识和属灵领悟力迟钝的事。这意味着我们对基督之事的属灵觉知和领悟力被这世界的事物迟钝、败坏、阻滞、瘫痪了。

如果我们在这世界中有我们的生命(我们不是指肉身,也不是因着在人群中活动的义务;我们指内在的生命是在这世界中)或与这世界有心系之处,那么我们就会在它的黑暗中行走。这就是为什么主说与祂的联合就是光,除非我们与祂一同出去,否则我们就会被留在黑暗中行走:「趁着有光,信从这光……」。祂并不是对门徒说他们会被撇下没有光;他们将能够在黑暗的世界中一直行在光中,因为他们与祂在天上有属天的联合。但如果我们被留在这个世界而没有那种联合,那就是黑暗。在灵里与这世界有任何接触都会使我们的属灵官能瘫痪。如果我们想要明白属天的事,那么我们必须有属天的生命,我们对主之事的领悟完全取决于我们在多大程度上活在与主自己的属天联合中。

\section{试验的地位}

还有一件事。我们的地位总是我们受试验的基础;也就是说,我们是受我们所生活的地位试验的。这在以色列人的例子中为我们说明了。他们已经习惯了在埃及的某种生活。那可能是一种艰苦的生活,可能有困难,但也有一些不太困难的事情。他们确实知道他们的食物从哪里来;他们的生活是一种凭眼见的生活。那是相当确定和规律的,在信心的事上并没有太多试验。他们相当清楚他们会过怎样的生活,那是一种相当固定的常规,没有什么不确定性;因此在信心上并没有太多试验。但当他们来到旷野时,他们发现自己被所处的地位大大试验。

试验常常对他们来说太过分了,他们就回想起埃及。「在这里我们对任何事情都没有把握,我们不知道下一顿饭从哪里来;这里的一切都不确定,似乎如此模糊,我们看不见,也做不了什么。我们自己所做的一切都无法确保我们所需的。我们只是在这里,完全依靠神!」啊,是的,这取决于你如何说,这是什么意思。你可以用一种暗示这是可怕之事的语气来说,或者你可以说:「好吧,我有最可靠、最确定的源头——我有神」。但肉体并不这样看事情。属血气的人认为依靠神而不依靠自己的努力、知识、智慧、力量和领悟是一件可怕的事。对属血气的人来说,处在一个自己的领悟、能力、力量和整个自己都完全无用,完全依靠自己以外某位者的境地,是一个可怕的地位。

我们被带出世界进入与主的属天联合,在那里一切都由祂而出,不是由我们自己而出,也不是由世界而出。这是一种完全依靠的生命,但这是何等的试验!你会发现你什么都做不了。在你自己里面没有任何东西可以应付这种情况。门徒们只要还能就紧紧抓住主,要把祂留在他们感官的领域内,因为他们觉得如果祂回到父那里,他们的世界就会崩溃。

教会真正的见证正是朝着这个方向,就是基督在天上的绝对真实性,由教会的生命所彰显。你从使徒行传开始,一直往下看,教会的见证就是见证耶稣虽然在天上,却比我们从前在祂肉身日子在地上所认识的更加真实、奇妙和荣耀!教会正是在那个基础上被赋予这个见证。看看使徒们的例子,再看看伟大的使徒保罗的例子,你就会看到教会作为他们的代表,其全部见证就是见证升天基督的奇妙真实性,复活之主的真实性。这一切都得到了确保。 主正试图向门徒们阐明,祂去是对他们有益的,因为他们将在祂去后发现关于祂的事,这是他们以前从未能够发现的。

在约翰福音13章和使徒行传9章之间发生了何等的变化。在约翰福音12章中,主耶稣处于羞辱中,被拒绝,被藐视。使徒行传9章揭示了来自天上的光,比日头还亮,带着一切权柄和能力在祂手中,以至于大数的扫罗在升天的基督面前遇到了远超过他能力的对手,而整个犹太教体系——集中体现在那个人身上——在升天的基督面前也遇到了远超过它能力的对手。在改变像大数的扫罗这样一个人——从他所持守的立场转变为被藐视之耶稣的忠诚仆人——这件事上,宇宙中没有任何其他事物能够做到的,却是藉着耶稣在天上的启示成就了。正是这件事震惊并击垮了他,正是这件事成就了这个神迹。「我是耶稣。」我们无法想象这个宣告对大数的扫罗产生了何等的冲击。

扫罗曾认为拿撒勒人耶稣是个骗子、假教师、亵渎神的人,并且死在神的咒诅之下,因为「凡挂在木头上都是被咒诅的」。对大数的扫罗来说,拿撒勒人耶稣是最可鄙、最可恨的对象。现在听听这出自超越荣耀、无法承受、令人目眩的宣告——「我是耶稣」!

教会在这一价值中有其见证,其职事完全关乎主耶稣是谁,不是在历史意义上,而是在祂属天、得荣耀、被高举的位格中。正是在神右边的人子赐给教会职事,神在荣耀中得着一个人这是什么意思。这就是教会的职事。

你看这如何与我们开头所说的联系起来。神永远的旨意是要藉着一个创造来在宇宙中彰显祂自己,这创造的中心是人,而这团体人(corporate man)的中心是人子基督耶稣。对世界和对撒旦的权柄——只有并且确实是在与基督属天联合的基础上。

那么你注意到,在第36节中,随「后来」这个词带进了如此多的内容。当基督在天上时,他们就会明白;当基督在天上时,他们就能够跟随。主对彼得说「你现在不能跟我去,后来却要跟我去」的意思不是「你现在不能跟我去,但当你死的时候你会和我一起去天上」,祂的意思是:「彼得,你在肉体中,而你的肉体是非常自信的肉体。你说你甚至愿意为我死。你不知道你的肉体要带你经过是多么不可能!肉体无法经过到底。但当你不再在肉体中而在灵里时,当圣灵降临,你成为一个属灵的人,而不是属肉体或属血气的人时,那时你就能跟随到底,你就能经历十字架,你就能为我而死,但现在不行。」这「后来」取决于基督在天上;我们理解的程度和我们能走多远,都取决于在圣灵大能中与基督的属天联合。

因此你看,所有的书信都是为了将属天的丰满带入圣徒里,也将圣徒带入属天的丰满中。这就是职事,这就是约翰福音13章的服事。它是要将主的百姓带入祂的属天丰满中,并将祂的属天丰满带入他们里面。

现在我们接近洗脚的事了。关于这一点有太多的误解。你必须从天上来看待事物。如果你开始从地上来看待事物,你就会一直走偏。

这是什么意思?为什么要洗脚?看哪,他们曾与地接触,他们与地的接触使他们污秽了。那么,现在就要摆脱一切属地的东西,保持脱离属世的、在属灵和道德意义上的世界。主说:「我要离开世界。你们必须在灵里与我一同离开世界,但你们仍要留在这里。你们的生命会在这里,虽然你们不属这世界。当你们在这世界中行动时,你们会被它沾染,但要记住你们是属天的子民,你们必须维持你们的属天关系。那么,你们彼此之间的职事就是要帮助彼此保持洁净,脱离世界,以帮助彼此住在属天的生命中。」这就是洗脚。这是一种相互的承诺,帮助彼此在这世界的事上保持洁净。

主耶稣在祂伟大的行动中实际上是在说:「我使你们成为一班人,一个团体,从这世界中洁净出来,成为属天的子民,但你们必须留在世界中作见证。你们会与世界接触,时不时它会把手放在你们身上,会有污秽,但不要停留在那里,不要让它留在你们身上。在你们的交通中,寻求帮助彼此保持自由,提升彼此脱离世界,把世界排除在外。如果你看到一位弟兄、一位姊妹被世界的一些事物所沾染,就寻求以爱心、谦卑的方式服事那人,使他们脱离那种沾染,那种定罪。」这正是祂后来所说的新命令的意思,就是你们「彼此相爱,正如我爱了你们一样。」「我如何爱了你们?我爱你们,要把你们从定罪和审判的范围中带出来。我爱你们,甚至我的生命为你们舍了,要把你们带出来成为属天的子民。现在你们必须在同样的爱里彼此服事,以维持那属天的地位,那属天的生命。」这是一种爱的职事。这不是官方的。这是一种彼此相爱的职事,以帮助走在属天的道路中(第35节)。

爱是倒空自己。祂脱下外衣,拿一条手巾束腰,把水倒在盆里,出来洗门徒的脚。这就是神儿子的倒空自己。如我们所知,这是腓立比书2章的预表:「既有人的样子,就自己卑微……取了奴仆的形象……存心顺服,以至于死,且死在十字架上。」这正好与那寻求辖制、作主的撒旦式骄傲相反。骄傲决不会洗别人的脚,它不会出来服事他人。骄傲总是在寻求被人服事。

主耶稣说:「如果你我真的要成为属天之事的真实执事,如果我们自己要进入那认识,并且要被主使用,我们就必须没有骄傲,我们必须以爱为特征。」而这种爱要以这种方式显明:我们不断寻求帮助彼此逃脱这个邪恶世界的网罗、沾染和污染,并使彼此能够活出属天的生命,与主属天联合的生命。这始终是提升,是高举。做相反的事是如此容易,题醒彼此关于我们周围旧造的事物,这会使我们降到地上。这没有多大帮助。如果你不断地题醒我仍然围绕着我的旧造事物,你就不会把我提升得很高。如果我一直指出仍然缠绕着你的旧过错,我也不会帮助你达到很高的境界。让我们彼此服事,叫人高升,进入生命的天性里。没有什么比不断地题醒彼此更提升人了:重要的不是我们是谁,而是祂是谁,祂在荣耀中为我们是谁。这立刻就是提升的事,正如霍拉修斯·博纳(Horatius Bonar)在他的诗歌中所说:「主啊,不是我是什么,而是你是什么。」这就是属天;这就是属天的职事,保持基督在视野中并服事基督。这才是真正的爱,这就是基督对我们的爱。

现在我们可以更多地看到主对祂百姓的心意,以及祂要达到目的的道路;因为你我和所有主的百姓只有从这世界中出来,成为属天的子民,活在凭借属天资源的属天生命中,在那里没有定罪,没有审判,也没有这世界的权势,而是处在神爱子的国中,才能达到那伟大的目的,即神藉着一个新造在宇宙中彰显。这就是通往荣耀的道路。

注意主耶稣将这两件事联系起来:他即将离去,现在人子得了荣耀。荣耀与从这邪恶世界中的拯救和属天的生命紧密相连。


\chapter{神全面的旨意}

\scripture{\textbf{约翰福音 15:1-27} 1 「我是真葡萄树,我父是栽培的人。2 凡属我不结果子的枝子,他就剪去;凡结果子的,他就修理干净,使枝子结果子更多。3 现在你们因我讲给你们的道,已经干净了。4 你们要常在我里面,我也常在你们里面。枝子若不常在葡萄树上,自己就不能结果子;你们若不常在我里面,也是这样。5 我是葡萄树,你们是枝子;常在我里面的,我也常在他里面,这人就多结果子;因为离了我,你们就不能做什么。6 人若不常在我里面,就像枝子丢在外面枯干,人拾起来,扔在火里烧了。7 你们若常在我里面,我的话也常在你们里面;凡你们所愿意的,祈求就给你们成就。8 你们多结果子,我父就因此得荣耀,你们也就是我的门徒了。9 我爱你们,正如父爱我一样;你们要常在我的爱里。10 你们若遵守我的命令,就常在我的爱里;正如我遵守了我父的命令,常在他的爱里。11 这些事我已经对你们说了,是要叫我的喜乐存在你们心里,并叫你们的喜乐可以满足。12 你们要彼此相爱,像我爱你们一样,这就是我的命令。13 人为朋友舍命,人的爱心没有比这个大的。14 你们若遵行我所吩咐的,就是我的朋友了。15 以后我不再称你们为仆人,因仆人不知道主人所做的事;我乃称你们为朋友,因我从我父所听见的,已经都告诉你们了。16 不是你们拣选了我,是我拣选了你们;并且分派你们去结果子,叫你们的果子常存,使你们奉我的名,无论向父求什么,他就赐给你们。17 我这样吩咐你们,是要叫你们彼此相爱。18 世人若恨你们,你们知道(或作『该知道』)恨你们以先已经恨我了。19 你们若属世界,世界必爱属自己的;只因你们不属世界,乃是我从世界中拣选了你们,所以世界就恨你们。20 你们要记念我从前对你们所说的话:『仆人不能大于主人。』他们若逼迫了我,也要逼迫你们;若遵守了我的话,也要遵守你们的话。21 但他们因我的名要向你们行这一切的事,因为他们不认识那差我来的。22 我若没有来教训他们,他们就没有罪;但如今他们的罪无可推诿了。23 恨我的,也恨我的父。24 我若没有在他们中间行过别人未曾行的事,他们就没有罪;但如今连我与我的父,他们也看见也恨恶了。25 这要应验他们律法上所写的话,说:『他们无故地恨我。』26 但我要从父那里差保惠师来,就是从父出来真理的圣灵;他来了,就要为我作见证。27 你们也要作见证,因为你们从起头就与我同在。}

实际上,第14、15和16章这三章构成了这一部分,其中包含了关乎神全部旨意的特殊真理。它们实际上虽是一个整体,不可分割,但第15章乃是核心,既回溯第14章,又延伸至第16章。

我们现在来到了启示最丰满的开展阶段,我们正触及那些最强有力之事,以及那些代表完整性与整体性的事。因此,回顾整个基础对我们是必要且有益的,以便看见这部分如何既融入其中,又在很大程度上将其汇集起来。

我们一直在思想神在祂儿子基督里永远的旨意,我们已经看见那旨意乃是一种表达、一种彰显、一种祂自己在创造中的启示;而在永远的计划中,那意念和旨意是要以人为中心的器皿和凭藉来实现并成全的,「人」在族类意义上的总和就是祂的儿子耶稣基督。

也许再次参考一些论及神永远计划的经文会很有价值,这些经文向我们展示了这些计划的内容。以弗所书是这一启示的主要管道。以弗所书1:4-6:「就如神从创立世界以前,在基督里拣选了我们,使我们在祂面前成为圣洁,无有瑕疵;又因爱我们,就按着自己意旨所喜悦的,预定我们藉着耶稣基督得儿子的名分,使祂荣耀的恩典得着称赞;这恩典是祂在爱子里所赐给我们的。」第11-12节:「我们也在祂里面得了基业,这原是那位随己意行作万事的,照着祂旨意所预定的,叫祂的荣耀从我们这首先在基督里有盼望的人可以得着称赞。」以弗所书2:7:「要将祂极丰富的恩典,就是祂在基督耶稣里向我们所施的恩慈,显明给后来的世代看」;第10节:「……我们原是祂的工作,在基督耶稣里造成的,为要叫我们行善,就是神所预备叫我们行的」;以弗所书3:9-11:「……又使众人都明白,这历代以来隐藏在创造万物之神里的奥秘是如何安排的,为要藉着教会使天上执政的、掌权的,现在得知神百般的智慧。这是照神从万世以前,在我们主基督耶稣里所定的旨意。」

这包含了几件别的事,但我们首先注意到的是,它代表了神在永世以前的计划。它向我们展示了一个事实:神在永远的计划中确定了某些事;计划、安排并确保了某些事。这些计划被总结在一个词组中:「永远的旨意」。

那么,在这些和其他经文中所看见的另外两件事,首先是那永远的旨意和那些永远的计划都与祂的儿子、主耶稣紧密相连,并且都在祂里面。第二件事是,作为神儿子的基督被展现为一位包容性和综合性的儿子,而一个伟大的群体,现在我们称之为教会——祂的身体——在那些永远的计划和预知中,乃是在基督里被拣选的。

因此你看见这个次序:神的永远计划,以及历代的旨意,总结在祂儿子耶稣基督里,并在教会中实现和彰显。这一切都被总结在一个词中,那个词就是:「儿子」。这是一个非常全面的词。

希伯来书开宗明义地指出:「神既在古时藉着众先知多次多方地晓谕列祖……就在这末世藉着儿子晓谕我们(边注作『一个儿子』,但希腊文没有对应『祂的』或『一个』的词,希腊文直译就是:『藉着儿子晓谕我们』)。当然,我们知道从紧接的下文来看,这首先是指主耶稣,『又早已立祂为承受万有的,也曾藉着祂创造诸世界(或诸世代)。』『子化』(Sonwise)这个词,正如这封书信很快继续表明的,既是一个包容性的词,也是一个位格性的词,而且这儿子被看作是与其他众子或被称为众子的人建立了团体关系,他们要被带进荣耀里。

如果我们再看歌罗西书第1章,我们会看见一个非常完整的呈现。第12-19节:「感谢父,叫我们能与众圣徒在光明中同得基业(这些都是家庭用语)……祂又救我们脱离黑暗的权势,把我们迁到祂爱子的国里;我们在爱子里得蒙救赎……万有都是靠祂造的……万有也靠祂而立……因为父喜欢叫一切的丰盛在祂里面居住。」

这里有特别与儿子相关的永远计划,然后我们的地位在儿子中被看见,与那些永远的计划相关联并紧密相连;也就是说,教会的地位。

因此,从那些计划的形成和投射到最终实现的整个过程中,支配一切的词、观念、思想就是「儿子的名分」(sonship),我们已经看见并很清楚地知道,儿子的名分是神在与祂自己关系这件事上的完整思想。这不是与『孩子』这个词相关的初始思想,孩子是一回事,但完整的观念是『儿子』,而这就是永远计划所要达到的终点。神的每一项行动总是以此为视角的:创造、救赎,万事——没有人能一一列举这一切,我们无法理解这意味着什么——祂照着自己旨意的计划在工作。因此,正在进行的一切都被神这唯一的目的所支配,即在这个宇宙中产生儿子的名分,按其全部意义,『使我们成为祂荣耀的赞美』;这就是终局,因此如果我们向前看,我们会看见万事的终局是一个充满对祂荣耀赞美的宇宙。

神在宇宙中、在自然界、在创造中、甚至在受造之物的叹息中(根据罗马书8章),以及在祂为我们这些按祂旨意被召之人的生命中所进行的那些微小活动里,都被祂与儿子的名分联系在一起。如果这真的抓住了我们的心并一直持定,我们对一切事物的态度就会是一种探究,探究这如何能导致儿子的名分,这如何能导致那种成长和发展,意味着神思想的完全表达。

我们应该确定下来,把它作为一个确定的、持久的、不可改变的事实,即从神的角度来看,一切都被管制着,以达到这个目的:就我们而言,就教会而言,要产生儿子的名分——按其全部意义。这是一个全面的思想。在这些日子的末后,不再是零碎的、以不同的方式,不再是在列祖中,而是现在一切都汇集并包含在一位里面,即子化(Sonwise)。

现在,我们是否重新看见了神的旨意,祂所要达到的目标?我们是否清楚地把它摆在我们面前?(啊,这是何等丰盛的宇宙!)从起初,神就按着儿子名分的原则行事。如果我们以此为视角,我们就能够回来理解约翰福音。事实上,我们能够理解所有的经文,但既然这卷福音书在此时特别带给我们,它揭示了儿子名分的伟大奥秘,并向我们展示了神如何以及藉着什么方式朝着那目标前行;即支配儿子名分的法则和原则。约翰写这卷福音书的全部目的,正如他在结尾所说的,是要证明耶稣基督的儿子名分。

现在,如果我们在儿子的名分中有分,那么支配祂生命的那些原则也支配着我们的生命,因此这福音书不仅是对耶稣基督的启示,也是对我们来说在基督里之事的启示,使我们也能够达到我们被预定得儿子名分的目的。

\section{约翰福音概要}

约翰福音第1章是从永远里对儿子的包容性和全面性呈现,而影响我们并立即影响神旨意的伟大陈述是这句话:「在祂里面有生命,这生命就是人的光。」这是一个呈现独特事实的陈述;在祂里面有生命,在其他任何地方都没有。然后它呈现了一个相关的真理:「这生命就是人的光。」祂与人作为光相关联。

第2章将丰盛的原则摆在我们面前。我们指的是加利利迦拿的婚宴。这将丰盛的原则摆在我们面前,因为那个表号性神迹的终局是处处丰盛;生命的丰盛,喜乐的丰盛,祝福的丰盛,荣耀的丰盛。好酒留到最后,祂显出了祂的荣耀。这正是达到的终局,如果你研究那里发生之事的细节,你会得到导致丰盛的原则。

现在,神总是在早期、在开始时就把祂的整个主题摆在眼前,然后祂才在之后分解并应用它。因此在第1章中,你在基督里有全面的呈现;在第2章中,你在基督里有丰盛原则的全面陈述。我们可以把它分解成片段。

在第3章与尼哥底母的对话中,指出了一个新造的必须,拥有属血气的人(natural man)所没有的新生命,如果要达到神的终局的话。在实现这些永远计划的第一步都无法迈出,直到在基督里有一个新造,这个新造以新生命为特征;不是从肉体生的,不是从情欲生的,也不是从人意生的;而是从圣灵生的,一个以神圣生命为特征的新造。

第4章向我们展示了下一步,那就是当那生命被植入信徒存在的中心时,它回应了内心对目的的深切探询。撒玛利亚妇人在各方面都以渴望、欲望、需要为标志。(「渴」是与人和神关系相关的一个特征性词汇)。这里有一个完全体现了自觉需要、渴望和想要知道真实生命的人,生命的意义是什么,一切的意义是什么——尽管可能没有用如此技术性的术语表达。有一种意识,认为应该有某种东西,但现在却没有。内心深处有一种知识,知道我们是为某种目的而造的,但我们找不到;这种生命是一种嘲弄,这种生命只是不断地寻求满足却永远找不到。

渴望被满足的欲望本身,以及需要的感觉本身,肯定见证了我们是为某种目的而造的事实。有一个我们被创造的命运,我们在摸索寻找它,但我们似乎从未找到或达到它。如果我们在地上结束生命时还是这样,那么生命就嘲弄了我们。这就是那妇人所代表的状态,然后主耶稣谈到祂所赐的水,神的圣灵的生命住在里面,表明这一切都有答案,那住在里面的神圣生命就是问题的解答,是整个情况的钥匙。当你拥有那生命时,你就拥有了命运的本质,你找到了你整个存在所渴望的东西,作为你存在本身的解释。认识到这一点是一件奇妙的事。我相信这并不太抽象。

你问:「为什么我有存在?我意识到渴望、欲望、向外伸展,如果它们没有得到完全和最终的满足,那么生命就嘲弄了我,我错过了某种我意识到我应该拥有的东西来解释我的存在。」现在,当主将永远的生命植入里面时,就有对整个情况的答案,那生命本身就说:「这就是你被造的目的;这就是你被创造的目的,为了神。」因此,在接受永远生命的那一刻,我们在本质上就拥有了神为我们创造的全部旨意,以及所有对那些标记我们得儿子名分的永远计划的答案。「因为你们是儿子,神就差祂儿子的灵进入你们的心,呼叫:『阿爸,父!』」那在基督里的生命之灵将我们与神以及神在我们创造中的旨意联系起来。

第5章和第6章引导我们看到与基督的联合是神圣旨意的完整基础和道路。我们不会详细停留在这一点上,但这两章非常清楚地阐明了神圣联合的事实。第5章强调基督与父的联合,以及它是如何运作的,祂对父的依靠,祂从父那里汲取一切,祂做一切事,说一切话都是从父而出,没有任何从祂自己而出的。然后祂将这一点应用到我们身上,表明正如祂靠父活着,我们也必须靠祂活着。第6章展示了与基督联合的形式,就是以祂为粮:「我是生命的粮」。祂靠父活着;我们必须靠祂活着。但第6章也强调与祂的联合是藉着死与复活,因为是擘开的饼。是祂的身体,祂的肉,和祂的血。这些在被释放之前是无法领受的——身体被擘开,血流出来,我们在灵里领受。因此与基督的联合是神圣旨意的完整基础和道路;也就是说,没有任何事情是在基督之外的。神不会把永远的生命作为某样东西赐给我们。我们只有在儿子里面才有它。

第7章引导我们进入与基督联合的属天性质。住棚节支配着那一部分,表明神的百姓被呼召出来住在帐棚里,并在整个地上的行程中维持那见证。他们的生命不是在地上有固定居所的属地生命,而是他们必须一直维持见证,表明他们不属这个世界;他们住在帐棚里;他们是属天的子民。这是神旨意的一个伟大原则,我们应该认识到我们是属天的子民,必须活出属天的生命。

第8章和第9章处理永远旨意的另一个伟大原则和法则;就是关于基督的启示是永远旨意中的一个至高因素。在第8章中,有很多关于主作为光的论述。在第9章中,你有生来瞎眼的人。这两件事相辅相成,其结果是对神儿子的完全认识;当然,是以预表的方式。那人最终看见了耶稣是谁,这就是光的运行结果。这就是保罗在以弗所书中所说的「智慧和启示的灵,使你们真知道祂」的意思;不仅仅是对基督的初步认识,而是对祂的完全认识。对耶稣基督的启示是永远旨意的法则;也就是说,我们藉着在我们里面对神儿子的完全启示,朝着神的儿子名分的终局前进。这是一个渐进的事情,藉着圣灵的启示对耶稣基督的不断增长和不断更新的认识,属灵的成长达到丰盛,达到完全,达到儿子的名分。

第10章将这一切汇集到一个团体中,并引入了这样一个事实:这不是许多独立单元的问题。神永远旨意的实现需要教会,并且旨在藉着教会来表达和彰显。因此,随着教会被引入,或在第10章中与基督联合的团体,我们被引导进入第11章和第12章,看见教会作为一个复活的团体。

第13章更进一步,展示了教会在属天的联合、属天的生命和属天的服事中。基督正离开世界,虽然在某种意义上祂把教会留在后面,但在另一种意义上,祂带着它一起去。这就是我们马上要看见的,因为洗脚说到持续地从世界中分别出来,所有的服事都是为了帮助主的百姓维持他们的属天地位和属天生命,不被这个世界缠累。

\section{第14、15和16章}

那么我们来到第14、15和16章。第14章向我们展示基督离开世界,或者主要强调的是这一点。第15章引进了一些对属血气的人(natural man)和这些尚未领受圣灵的门徒来说,总是令人费解、看似矛盾之事。主耶稣一再说:「我要离开世界;我要到父那里去;等不多时,你们就不得见我了」——「你们要常在我里面」。属血气的人说:「这怎么可能?」这正是门徒的困难,「你说你要去;我们不明白你说『等不多时』是什么意思。这已经是个问题了,但除此之外,你还说我们要常在你里面,而你要去。你在谈论一个我们无法理解的领域。」嗯,当然,属血气的人不可能理解这样的事。你我和他们不同,因为我们活在其中的真理和实际中,但我们立刻可以看出这如何与永远的旨意相关。基督已经去了;基督在天上,既然在那里,教会关于神永远旨意的一切都随祂转移到天上。这是在世界之外。有一个新的地方,那里有一个人,那个人和那个地方是一切与教会相关的神圣旨意的源头、泉源。在天上的基督是一切与神旨意相关的圣徒的范围和源头。

祂说,如果要达到或实现那旨意,就必须常在祂里面,作为在世界之外、在天上的人,常在祂里面作为万有的泉源。

以弗所书中有一段经文是对这一点更完整、教义性、属灵性的阐述:「神……使祂……为教会作万有之首,教会是祂的身体,是那充满万有者所充满的。」教会的泉源在天上,对万有来说,常在祂里面是必需的。

门徒的问题,以及如果我们知道的话也会是我们的问题是:「在地上的人如何能常在天上的一位里面,以及那在天上被藏在祂里面的丰盛如何能释放给在地上的人,以便达到神在儿子名分中的终局?」

第16章主要回答了这个问题,但这三章都涉及这一点。答案是圣灵。常在基督里只是另一种说法,即活在圣灵里,靠圣灵而活,让你的整个生命在圣灵里,并从圣灵汲取你所有的资源。我们知道这一点,但这是一个必须不断重复的事,我们必须不断地被题醒。儿子的名分终究是一件让你的生命在圣灵里,并让圣灵的生命在你里面,并且是充满的事。当我们使用「儿子的名分」这个词时,我们只是用一个词包含了神的所有永远计划,也就是历代的旨意所指的。

现在你理解了约翰福音15章,你可以分解这一章。一方面,主耶稣说,若不常在祂里面,就不可能有任何事情。在某个领域里很多事情是可能的,但在神旨意的领域里却不是。你可能会做很多事情,成就很多事情,但它们是在永远旨意的领域之外。它们与儿子的名分无关,不会导致儿子的名分,这才是重要的事。

但当涉及到神的终局时,那么重压在整个情况上的一个词就是:若不常在基督里,就不可能有任何事情。换句话说,若不活在圣灵里,就不可能有任何事情。「……离了我,你们就不能做什么」。另一方面,积极的一面是,藉着常在基督里,一切皆有可能。让这进入我们的心并带来安息。它这样说:神计划了一个伟大的终局;神定意了非凡的事。这一切意味着你和我身上必须发生巨大的改变。我们必须成为不同的受造之物;我们必须效法神儿子的形像。一切都必须像基督,属于基督。

这代表着巨大的转变。这是一个非常高的标准。这意味着完全。然后我们在心中软弱地说:「这怎么可能?这太过了,太高了,超出了我们的能力。」我们如此意识到在我们里面与那对立的巨大数量,邪恶的积极力量,以及良善的绝对软弱。怎么办?答案是:只要你常在,这一切都会发生。就是这样。你不必去做。你不必去产生。你不必使自己效法神儿子的形像。你不必创造出要归给神荣耀的果子。你根本不必去产生这一切。你所要做的就是常在基督里。活在圣灵里,这一切都会发生。这就是约翰福音15章的意思。

神全面旨意的一切都在常在之中。这理当带来安息,这正是基督成为我们的安息日,我们的安息。圣灵会做一切,只要我们常在祂里面,当我们常在时,果子就结出来了。

\section{果子的定义}

让我们非常清楚果子是什么。不要混淆事情。很多人认为约翰福音15章的果子是他们为主所做的工作,以及那工作的结果。不是的。在另一种意义上,生命的丰盛可能体现在得人灵魂和帮助他人归向基督上,但这不是这里的果子。这里的果子是实现神的终局和祂自己的彰显:「你们多结果子,我父就因此得荣耀……使我们成为祂荣耀的赞美」。那么果子是什么呢?就是表达神,彰显神。「圣灵所结的果子就是仁爱、喜乐、和平、忍耐、恩慈、良善、信实、温柔、节制……」。这是神的表达。

如此多的人持有这种观念,认为根据约翰福音15章,结果子是我们在为主做事,他们就出去为主做工,试图藉着做事来结果子。他们未能看见,如果你我常在基督里,或常在圣灵里,那种事情就不会是我们所采取的。它会变得自发。你看灵魂,你说:「我必须得着这些灵魂!」于是你就在一种律法的鞭策下,直接冲向每个人。然而,如果你常在基督里,警醒并预备好(当然常在基督里意味着在这些事情上不常在我们自己里面,也不常在我们对事情的感受里面),圣灵知道一个灵魂何时预备好,何时未预备好。时机是祂所知道的;祂自己安排这些事情的时机。在人的生命背后有主权的运行,他们自己并不知道,当主最终来到一个生命时,正好是在正确的时候。

那个埃塞俄比亚太监正好成熟了,圣灵知道并抓住腓利说:「去,贴近那车;这里有一个人预备好了,这是那人的时刻。」圣灵知道哥尼流并对彼得说话:「这里有一个人预备好了;时候到了。」圣灵知道。我们另一方面,拿起这件事,试图结果子。

没有人应该认为我们在劝阻对灵魂的热切渴望。这是圣灵在我们里面工作的一部分,我们应该有热切的渴望,但如果我们出去拿起这件事说:「我必须结果子,得人灵魂」,并仅仅从我们自己里面去做,可能会有很多灰心和很多问题迟早会出现,而圣灵自发的工作总是有果效的。常在基督里是为了各种果子。

我们想要强调的是:在其他生命中有那些实际、具体的丰盛表达之前,必须先有主在我们里面的表达。职事出自我们是什么,完全取决于我们是什么,当我们的职事变得超前于我们与神同行时,我们就进入了不育(barrenness)、不结果子、无效的领域,迟早会失败。所有的职事必须与我们与神的同行一致。常在祂里面,就结果子。就是这条路。让你的生命在圣灵里,神的终局就会达到;也就是说,祂自己的表达,儿子的名分;并藉着祂在我们里面,其他人也会被遇见,因为重要的不是我们能对别人说什么关于主的事,而是我们能给别人什么关于主的事。换句话说:圣灵在我们里面有什么基督可以给予,「我是生命的粮」,祂放在门徒手中,说:「你们给他们」。祂把自己给他们作为表号,让他们给别人。这就是职事。是领受基督,给予基督,不是谈论祂。为了那职事我们必须常在,当我们常在时,那就会发生。挣扎、紧张和压力从生命中消失了,对主的工作和我们自己属灵成长的烦躁和焦虑也会离去。当我们常在时,安息就来了。我们的工作就是常在,活在圣灵里并靠圣灵而行。其余的一切都会自发地产生,我们可以就此放手。

让我们有那样的理解,并非常清楚地说:「现在,主,我的渴望和意图就是单单常在你里面。其余的一切都是你的事。我的属灵成长是你的事。我的职事,我的服事,是你的事。我常在!」听起来好像太简单,太容易了;我们想要做点什么。这正是问题所在,我们又回到自己里面了。

正是藉着圣灵里的生命,我们才能达到神的终局,以及儿子的名分所意味的一切;也就是说,祂自己的表达,带着所有荣耀的结果,首要的是成为祂荣耀的赞美。

\chapter{通过儿子名分实现神的旨意}

\scripture{\textbf{约翰福音 17:1-26} 1 耶稣说了这些话,就举目望天,说:「父啊,时候到了;愿你荣耀你的儿子,使儿子也荣耀你,2 正如你曾赐给他权柄管理凡有血气的,叫他将永生赐给你所赐给他的人。3 认识你——独一的真神,并且认识你所差来的耶稣基督,这就是永生。4 我在地上已经荣耀你,你所托付我的事,我已成全了。5 父啊,现在求你使我同你享荣耀,就是未有世界以先,我同你所有的荣耀。6 你从世上赐给我的人,我已将你的名显明与他们。他们本是你的,你将他们赐给我,他们也遵守了你的道。7 如今他们知道,凡你所赐给我的,都是从你那里来的;8 因为你所赐给我的道,我已经赐给他们;他们也领受了,又确实知道,我是从你出来的,并且信你差了我来。9 我为他们祈求;不为世人祈求,却为你所赐给我的人祈求,因他们本是你的。10 凡是我的,都是你的;你的也是我的;并且我因他们得了荣耀。11 从今以后,我不在世上,他们却在世上;我往你那里去。圣父啊,求你因你所赐给我的名保守他们,叫他们合而为一,像我们一样。12 我与他们同在的时候,因你所赐给我的名保守了他们,我也护卫了他们;其中除了那灭亡之子,没有一个灭亡的,好叫经上的话得应验。13 现在我往你那里去;我还在世上说这话,是叫他们心里充满我的喜乐。14 我已将你的道赐给他们;世界又恨他们,因为他们不属世界,正如我不属世界一样。15 我不求你叫他们离开世界,只求你保守他们脱离那恶者。16 他们不属世界,正如我不属世界一样。17 求你用真理使他们成圣;你的道就是真理。18 你怎样差我到世上,我也照样差他们到世上。19 我为他们的缘故,自己分别为圣,叫他们也因真理成圣。20 我不但为这些人祈求,也为那些因他们的话信我的人祈求,21 使他们都合而为一;正如你父在我里面,我在你里面,使他们也在我们里面,叫世人可以信你差了我来。22 你所赐给我的荣耀,我已赐给他们,使他们合而为一,像我们合而为一。23 我在他们里面,你在我里面,使他们完完全全地合而为一,叫世人知道你差了我来,也知道你爱他们如同爱我一样。24 父啊,我在哪里,愿你所赐给我的人也同我在那里,叫他们看见你所赐给我的荣耀;因为创立世界以前,你已经爱我了。25 公义的父啊,世人未曾认识你,我却认识你;这些人也知道你差了我来。26 我已将你的名指示他们,还要指示他们,使你所爱我的爱在他们里面,我也在他们里面。」

    \textbf{民数记 3:5-7} 5 耶和华晓谕摩西说:6 「你使利未支派近前来,站在祭司亚伦面前,好服事他。7 他们要替他和全会众在会幕前守职,办理帐幕的事。

    \textbf{民数记 3:7}  「他们要替他和全会众在会幕前守职,办理帐幕的事。

    \textbf{民数记 3:11-13} 11 耶和华晓谕摩西说:12 「我从以色列人中拣选了利未人,代替以色列人一切头生的;利未人要归我。13 因为凡头生的是我的;我在埃及地击杀一切头生的那日,就把以色列中一切头生的,无论是人是牲畜,都分别为圣归我;他们定要属我。我是耶和华。」

    \textbf{民数记 3:41} 你要将利未人归我,我是耶和华,代替以色列人中一切头生的,利未人的牲畜也要代替以色列人中一切头生的牲畜。」

    \textbf{民数记 3:45} 「你当取利未人代替以色列人中一切头生的,利未人的牲畜代替以色列人中一切头生的牲畜。利未人要归我;我是耶和华。

    \textbf{民数记 8:13} 你要使利未人站在亚伦和他儿子面前,作为摇祭献给耶和华。

    \textbf{民数记 8:15} 以后利未人要进去办会幕的事;你要洁净他们,当作摇祭奉上;

    \textbf{使徒行传 20:28} 圣灵立你们作全群的监督,你们就当为自己谨慎,也为全群谨慎,牧养神的教会,就是他用自己血所买来的。

    \textbf{以弗所书 5:25-26} 25 你们作丈夫的,要爱你们的妻子,正如基督爱教会,为教会舍己,26 要用水藉着道把教会洗净,成为圣洁,

    \textbf{希伯来书 12:23} 有名录在天上诸长子之会所共聚的总会,和审判众人的神,以及被成全之义人的灵魂,

    \textbf{希伯来书 2:10-12} 10 原来那为万物所属、为万物所本的,要领许多的儿子进荣耀里去,使救他们的元帅因受苦难得以完全,本是合宜的。11 因那使人成圣的和那些得以成圣的,都是出于一;所以他称他们为弟兄也不以为耻,12 说:『我要将你的名传与我的弟兄,在会中我要颂扬你。』

    \textbf{希伯来书 2:17} 所以,他凡事该与他的弟兄相同,为要在神的事上成为慈悲忠信的大祭司,为百姓的罪献上挽回祭。

    \textbf{希伯来书 3:1} 同蒙天召的圣洁弟兄啊,应当思想我们所认为使者、为大祭司的耶稣;}


要看出约翰福音十七章与我们所读之经文(无论在旧约或新约)之间的属灵关联,并非难事。这些经文乃是本章的注释与阐释。当我们在此默想中来到这一点时,我们将让这些经文为我们解明约翰福音十七章。

我们一直将神的终极目标摆在面前。唯有在神永恒旨意的光中,来看祂话语的任何部分,我们才能明白并从中获得全部的价值。故此我们再次提醒自己,神以预表性的方式陈明祂完整的意念,并定规那所陈明的意念要成为历世历代的支配性意念。神的完整意图或心意是要彰显祂自己,并在整个宇宙中给予自己一个表达;祂选择藉着一个器皿来实现此事,这器皿将成为神彰显的工具,因此那器皿必须从祂那里取得其特性。它不仅是神的反映,更是祂活生生的表达。这就是神计划作为万事之终局的。

\section{利未人的职事}

既是如此,利未人便代表了那意念。我们务须记住,旧约乃是一本属灵原则之书,不可按字句来理解,因为这些机会都被包裹在预表和象征之中。预表和象征本身总是不及它们所要预表的事物,我们必须寻找的乃是它们所要体现并陈明的属灵真理。因此,利未人体现了一个伟大的属灵原则,我们不能把利未人当作代表所有时代的字句事物(我们稍后会进一步谈到这一点)。

我们首先要通过预表的方式,即利未人确实体现并陈明了神的完整意念——也就是彰显祂自己,神表达自己,神使一个世界认识祂自己,神藉着一个渠道将对自己的认识传播出去。换句话说,这是一个器皿,一个站在那里向人、向宇宙展示神心意的器皿;因为这不仅是为了人,而且就教会的真理隐藏在利未人中而言,藉此对神的彰显也是为了执政的和掌权的。这是对神普世性的彰显,是神的表达。

认识到这是包容性与全面性的事实后,我们可以将其分解成较小的部分,并留意关于利未人的几件事。

首先,利未人始终当受尊荣、重视与关怀。你注意到即使在极其可怕的堕落日子,当事物远离了神的心意时,利未人仍受尊荣与尊重。他在百姓中占有一席之地——有时甚至是迷信地——但他仍被认可。

其次,注意利未人与其他百姓的关系。利未人作为一个支派被取来代替一切头生的。主说,在祂击杀埃及地头生的那日,祂取了以色列所有头生的归自己。在头生的里面,全家便视为被包括并总结在内。头生的以包容性和代表性的方式站在全家之上,因此,在取头生的同时,主说:「全家是我的,其余即或不然,也都属于我」,就像初熟的果子被神索要作为整个收成和所有果实的凭据,归神所有,为神保留。因此,当利未人被取来代替头生的时,从神的角度来看,这是取所有百姓归祂。故此他们代表了全以色列人在神面前的关系,作为一个祭司的国度。

这就是我们之前所说的,我们不能按字句来理解他们,而要在属灵意义上理解,视其为体现一个原则。利未人在属灵上并不代表一个分离的阶级。如今人们谈论圣品与平信徒,但这乃是与神话语相异的思想,正是因未能认识这一事实,才导致了如此的分裂和错误的地位:利未人不是主百姓中的圣品阶级,而是主百姓在服事中。主所有的百姓都被视为在祭司的职分中服事主。

再者,注意相应的真理,即利未人与神的关系。他们要作为摇祭献给耶和华。他们是祂的。他们以完全的方式属于祂:「他们是我的,耶和华说。」但他们是祂用血买赎的,赎回归神的。这建立在以色列历史最基础的逾越节羔羊之血上。这又以另一种象征性的方式继续,藉着半舍客勒银子;用半舍客勒银子赎回头生的。银子,如我们所知,是救赎的预表。因此利未人陈明了神的百姓因救赎、用宝血买赎而属于祂的事实。

接下来的事是利未人被赐给亚伦来服事他。他们是给亚伦的礼物,就是给大祭司的。我们几乎不可能在此刻关闭耳朵,不让约翰的话进入:「……你所赐给我的那些人……他们是你的,你将他们赐给我……」——但我们不能过早地预期事物。利未人被赐给亚伦,作为他在主见证上的同伴,负责主见证的保存、维持和延续。他们是亚伦的同伴。

这几件事帮助我们来到约翰福音十七章,因我们无疑是面对着大祭司,根据祂在这祷告中自己的话,祂正在分别为圣自己,或已经分别为圣自己。就主耶稣自身而言,本章一切尽是大祭司性质的。在这整卷福音书中,祂一直被陈明为一个接一个地取代犹太人的礼仪。祂取代了逾越节,并自己成为逾越节的羔羊,那流出的血。祂取代了住棚节。祂直接进入了所有犹太节期的地位,并将它们转移到自己身上。如今最后祂直接进入了大祭司的地位,也将其转移到自己身上。但祂并非独自一人。有一个祭司的群体是祂所关心的,祂将自己的分别为圣与他们的分别为圣联系起来:「我为他们的缘故,自己分别为圣,叫他们也因真理成圣。」有一个祭司的群体在大祭司主耶稣的分别为圣中被分别为圣。

现在注意本章中的家庭元素(这是事情的核心)。注意「父」这个名的重复出现:「公义的父」;「圣父」;「父啊,时候到了」。然后注意与此相连的「子」;「愿你荣耀你的儿子,使儿子也荣耀你」。然后,在那关系中:「他们本是你的,你将他们赐给我」;「你所赐给我的那些人」。


\section{儿子名分}


当你想要解释这一点时,你可以转向希伯来书的经文:「祂称他们为弟兄也不以为耻……说:我和神所赐给我的众子……我要将你的名传与我的弟兄……所以,同蒙天召的圣洁弟兄啊,应当思想我们所认为使者、为大祭司的耶稣」。这是一个大祭司的家庭,这带我们来到事情的核心。核心就是儿子名分。神从永恒以来决定藉什么方式、在什么基础上在整个宇宙中启示、彰显并表达自己?就是藉着儿子名分,以及儿子名分在其完整思想中的意义。「儿子式」乃是神对自己彰显的永恒思想。

主耶稣在本章中——虽然没有使用实际的字眼——清楚地被陈明为从永恒以来的儿子,那首生的。祂两次谈到祂与父在世界被造以前的关系,正如我们已经指出的,祂在创世以前就被拣选为儿子,在祂里面神要总结万有。然后在祂里面我们也被拣选并预定得儿子的名分。

现在我们理解两件事:利未人在原则上是首生的;以及希伯来书十二章22-23节中的话:「……你们乃是来到……有名录在天上诸长子之会所共聚的总会」。因此儿子名分乃是支配一切的,希伯来书再次告诉我们,因着我们与祂的关系,「儿女既同有血肉之体,祂也照样亲自成了血肉之体……那使人成圣的(圣别者)和那些得以成圣的(被圣别者),都是出于一,这些都是祂要领进荣耀里的许多儿子。

这将为我们奠定一切的基础。首先,它将向我们解释我们与基督联合的真实性质,由神所生,与基督一同复活,与祂在复活的生命中联合。

如果你愿意带着这个思想长时间默想约翰福音十七章,你会发现它无比有益。约翰福音的其余部分都汇集在第十七章中。主耶稣以言语和行为陈明了与神永恒终极目标相关的大原则,即藉着在基督里的一个群体在整个宇宙中彰显祂自己;并且,在以言语和行为陈明那些法则、那些原则之后,祂将它们全部汇集在祷告中,然后,可以说,祂将整件事祷告出来。

这是给我们的一课。原则不仅是抽象之物,乃是须要以此祷告之事。它们必须被祷告进入行动,进入实现,进入成全。似乎在圣灵的领域中,主耶稣拿起所有这些事并向父祷告。祂回到永恒的过去,从神的计划中拿起这些事,然后祂继续前行,触及所有这些与神预定的旨意和目标相关的事,即祂荣耀的彰显。祂在这一祷告中触及神在祂里面得荣耀,祂在神里面得荣耀,以及教会进入那荣耀。祂触及神的旨意只能在一个身体中实现的事实,分享一个生命,在一个圣灵的合一中,祂将这一切祷告出来。

这并非分散和无关个体的事;这是一个群体的事,祂将它祷告出来。因此祂从开始就触及所有这些事。

我们看到,支配神终局的第一个原则是拥有永恒的生命,就是新创造的生命,祂两次触及这一点:「……叫祂将永生赐给你所赐给祂的人。认识你——独一的真神,并且认识你所差来的耶稣基督,这就是永生。」原则都被拿起,神这伟大旨意的一切元素都被祷告并贯穿,汇集在祷告中。这就是你我和我们最终要达到的地步;不是在积累原则,不是在构成一本属灵法则的手册,而是在祷告中俯伏,让这些事被祷告进入表达,进入实现。

祂汇集了十六章的内容,就其支配性原则而言,并将它们贯穿祷告。这是一件令人印象深刻的事。祂以包容性和全面性的方式,将儿子名分的问题贯穿祷告。在神永恒的计划中,儿子们被预见、预定并早晚在祂里面被拣选,光有这还不够。这是一个伟大的真理,但即使这样也必须被祷告。这是一个奥秘。我们在心思上无法调和这些事。它们对我们来说将永远是一个奥秘,但事实是,虽然神从一开始就知晓一切,从永恒就看见终局并拥有终局,但那终局的实现乃是藉着祷告,藉着与祂的合作而发生的。祂指出了这一点。

儿子名分的整个问题,汇集在其所有元素和特征中,正在被祷告贯穿。你可以看到教会在诸长子的教会中是什么样子。它在起源上是属天的,在生命上是属天的,在交通上是属天的,在合一上是属天的,在使命上是属天的。这些都是在利未人的预表和象征中标记出来的。它的属天性来自于它接受了一个属天的生命这一事实,这种属天性藉着按照那属天的生命生活而发展,那生命一直从被高举的主、天上的主汲取生命和所有资源。因此它在其属天的本性和属天的使命中发展。它的一致性,它的合一,就是它藉着圣灵所拥有的一个生命的表达。

我不认为我们现今应当照着约翰福音十七章那样去祷告。我们经常听到人们用约翰福音十七章的措辞祷告,「使他们都合而为一」。那祷告已蒙应允。在五旬节,那祷告得到了应允。祂将它祷告出来,在五旬节得到了应允。教会本是一体的。我们不需要做任何事来制造基督身体的合一。我们所要做的只是持守它,看顾它,珍惜它,并殷勤保守它。来自各族、各方、各民、各国的每个信徒都藉着一个圣灵分享一个生命,并在那基本事实上有他们的合一。我们自发地发现这一点。当你遇见一个神的活孩子时,你不用任何介绍就知道。如果你遇见一个真正拥有那生命的人,不久你就会发现他们。那生命是一个生命;它是一同流动的。

仇敌取得了巨大的成功,将主的百姓降到一个接受破碎合一的水平,并认为他们必须努力恢复合一,而不是采取积极的态度,即无论在地上发生什么,在天上,在基督里身体是一体的,没有什么能改变这一点。就对它的见证而言,那是另一回事。那就是我们必须殷勤之处。

这是我们利未人的职事,作见证,维持对属天事实的见证。利未人是属天事物的代表。希伯来书告诉我们,这帐幕和与它有关的一切都是天上事物的样式。利未人是天上事物的样式。主渴望在这里有一个对属天实际的见证。

我们不敢采取相反的观点,即因为身体在天上是不可分割、不可破坏的一体,所以我们如何对待彼此在这里就无关紧要。这确实有关系。我们的属天职事和使命与在这里表达属天事物、神的心意有关,这就是利未人的职事,我们对此必须殷勤。

如果我们此时继续看利未人的职事是什么,我们会看到它与见证的帐幕(会幕)有关,那就是基督和祂的肢体;一个群体的事物。他们负责看守它。但我们想要保持这里的重点,我们就暂时停在这里。

儿子名分是神彰显自己的方式、手段与途径,儿子名分对神来说首先是一件已确立的事,但对我们来说是一件渐进的事。当然,所有神圣的真理都是如此。对神来说,它们是完整、完美和确定的。甚至教会对祂来说也是完成的,最后一个成员对祂来说已经添加;因此在以弗所书中它总是被表现为一件完整的事,但从我们的角度来看,这件事是渐进的。儿子名分的开始是当祂差遣祂儿子的灵进入我们的心,使我们呼叫:「阿爸,父!」那是婴孩的呼喊,是儿子名分之灵在其初始阶段的呼喊。但那儿子名分必须是渐进的,这是神关注和关怀的对象。「神待你们如同待儿子」,在祂与我们所有的活动中,祂始终专注于发展儿子名分所意味的一切,完全的身量,属灵的成熟;其终局将是对祂自己的最充分彰显。这既是个人的也是群体的事,虽然我们应该从认识到个人和个人方面以及神与我们交往的意义中得到一切帮助:祂对付我们的方式,祂带领我们经历的经验,我们生命中的奥秘,艰难的事,都是为了儿子名分。

但让我们记住,儿子名分在其最充分的表达中绝不可能藉着个人或任何数量的个人本身实现。它需要群体的身体,因此神藉着群体生命有一种特殊和特别的对付,这在分开的个人生命中是无法知道的;那就是神的家;神百姓的交通为神提供了一个发展儿子名分的特殊机会。对肉体来说,神的家是最困难的地方。

雅各来到伯特利,但他在肉体中,他是一个属血气的人,不可能留在那里。神的家对肉体来说总是一个可怕的地方,雅各说:「这地方何等可畏!」雅各继续前行,在二十年中他一直在神的管教之下,然后他回到伯特利,他能够留下来了。他现在适合神的家了,他可以在那里安息。

圣徒真正的属灵交通对我们来说是一个非常困难的地方,如果我们是在肉体中。我们发现摩擦、冲突、关系的试炼、主百姓不同体质、性情和气质的困难。啊,但这是活在圣灵中而不是肉体中的至高机会;是活在圣灵的基础上而不是天然生命的基础上。这是对一切旧造事物的至高超越机会。主百姓的交通是一种巨大的纪律和训练。对我们和神来说,没有什么比圣徒之间真正的属灵交通更有价值了,但正是在那里你找到你的训练,正是在那里你找到你的纪律,正是在那里你不断被要求不要活在肉体中,否则你会制造麻烦,而是要一直活在圣灵中。在那里你绝不能让你天然的感受兴起并影响你。你必须让神的爱一直得胜。

藉着聚会生活与交通,神确保了一些如果祂把我们作为孤立的个体放在世界不同地方就无法确保的东西。这就是生命,我们不断地拒绝接受人们天然的样子,而活在神的爱中,活在基督的灵中。我从未遇见过一个我完全找不到缺点的人;不是我们在寻找缺点,而是迟早我们会遇到一些我们不赞同的东西,我们认为是奇怪的、奇特的,或者他们没有会更好的东西。现在,这正是超越的机会。正是在祂百姓的群体生命中,主有特殊的机会。我相信我们的训练不可能完全,直到我们达到在群体生活中知道神恩典得胜的地步。难道这不是主工作中的悲剧吗——基督徒无法彼此相处,宣教士与同工争执,主的工作被拦阻,见证失落?

神的家是服事的伟大训练中心——我们的意思是神百姓的交通——没有人应该被放在属灵责任的位置上,除非他已经在交通中毕业,已经学会与主的百姓中难处的人凯旋地(triumphantly)共处。如果人们很难相处,辞职去其他地方工作很容易,那里的人不那么难相处。如果我们这样做,我们就抛弃了获得尊荣资格的至高机会。

因此主将这件事祷告出来,祂将见证直接放在天平上:「使他们都合而为一:正如你父在我里面,我在你里面,使他们也在我们里面:叫世人可以信」。这就是你的职事;这就是你的见证。神在我们的关系中以纪律的方式对付我们,以发展儿子名分。

也许在这一点之后,我们应该对人采取不同的态度,看到那位主没有从我们生命中移除的难处之人正是祂发展儿子名分的方式,当儿子名分被发展时,主或许能改变情况——或者情况可能已经因我们的改变而改变了。

儿子名分是神要藉此在宇宙中彰显自己的方式,这将发生在我们——尽可能地在个人方面,以及更大程度地在群体方面——效法神儿子的形象时,我们最终将达到整个宇宙所等待的儿子名分,整个受造之物都被服在虚空之下。那时,当儿子们被彰显时,受造之物将得释放。


\chapter{十字架与神永恒的旨意}

\scripture{\textbf{约翰福音 17} 1 耶稣说了这些话,就举目望天,说:「父啊,时候到了;愿你荣耀你的儿子,使儿子也荣耀你,2 正如你曾赐给他权柄管理凡有血气的,叫他将永生赐给你所赐给他的人。3 认识你——独一的真神,并且认识你所差来的耶稣基督,这就是永生。4 我在地上已经荣耀你,你所托付我的事,我已成全了。5 父啊,现在求你使我同你享荣耀,就是未有世界以先,我同你所有的荣耀。6 「我已将你的名显明与你从世上赐给我的人;他们本是你的,你将他们赐给我,他们也遵守了你的道。7 如今他们知道,凡你所赐给我的,都是从你那里来的;8 因为你所赐给我的道,我已经赐给他们;他们也领受了,又确实知道,我是从你出来的,并且信你差了我来。9 我为他们祈求;不为世人祈求,却为你所赐给我的人祈求,因他们本是你的;10 凡是我的都是你的,你的也是我的;并且我因他们得了荣耀。11 我不再在世上,他们却在世上,我往你那里去。圣父啊,求你因你所赐给我的名保守他们,叫他们合而为一,像我们一样。12 我与他们同在的时候,因你所赐给我的名保守了他们,我也护卫了他们;其中除了那灭亡之子,没有一个灭亡的,好叫经上的话得应验。13 现在我往你那里去;我还在世上说这话,是叫他们心里充满我的喜乐。14 我已将你的道赐给他们;世人恨他们,因为他们不属世界,正如我不属世界一样。15 我不求你叫他们离开世界,只求你保守他们脱离那恶者。16 他们不属世界,正如我不属世界一样。17 求你用真理使他们成圣;你的道就是真理。18 你怎样差我到世上,我也照样差他们到世上。19 我为他们的缘故,自己分别为圣,叫他们也因真理成圣。20 「我不但为这些人祈求,也为那些因他们的话信我的人祈求;21 使他们都合而为一;正如你父在我里面,我在你里面,使他们也在我们里面,叫世人可以信你差了我来。22 你所赐给我的荣耀,我已赐给他们,使他们合而为一,像我们合而为一;23 我在他们里面,你在我里面,使他们完完全全地合而为一,叫世人知道你差了我来,也知道你爱他们如同爱我一样。24 父啊,我在哪里,愿你所赐给我的人也同我在那里,叫他们看见你所赐给我的荣耀,因为你爱我,在创立世界以前就爱我。25 「公义的父啊,世人未曾认识你,我却认识你;这些人也知道你差了我来;26 我已将你的名指示他们,还要指示他们,使你所爱我的爱在他们里面,我也在他们里面。」

    \textbf{约翰福音 18} 1 耶稣说了这话,就同门徒出去,过了汲沦溪,在那里有一个园子,他和门徒进去了。2 卖耶稣的犹大也知道那地方,因为耶稣和门徒屡次在那里聚集。3 犹大领了一队兵和祭司长并法利赛人的差役,拿着灯笼、火把、兵器,来到园里。4 耶稣知道将要临到自己的一切事,就出来对他们说:「你们找谁?」5 他们回答说:「拿撒勒人耶稣。」耶稣说:「我就是。」卖他的犹大也同他们站在那里。6 耶稣一说「我就是」,他们就退后倒在地上。7 他又问他们说:「你们找谁?」他们说:「拿撒勒人耶稣。」8 耶稣说:「我已经告诉你们,我就是。你们若找我,就让这些人去吧!」9 这要应验耶稣从前说的话:「你所赐给我的人,我没有失落一个。」10 西门彼得带着一把刀,就拔出来,将大祭司的仆人砍了一刀,削掉他的右耳;那仆人名叫马勒古。11 耶稣就对彼得说:「收刀入鞘吧!我父所给我的那杯,我岂可不喝呢?」12 那队兵和千夫长并犹太人的差役就拿住耶稣,把他捆绑了,13 先带到亚那面前,因为亚那是当年作大祭司该亚法的岳父。14 这该亚法就是从前向犹太人发议论说「一个人替百姓死是有益的」那位。15 西门彼得跟着耶稣,还有一个门徒也跟着。那门徒是大祭司所认识的,他就同耶稣进了大祭司的院子,16 彼得却站在门外。大祭司所认识的那个门徒出来,和看门的使女说了一声,就领彼得进去。17 那看门的使女对彼得说:「你不也是这人的门徒吗?」他说:「我不是。」18 仆人和差役因为天冷,就生了炭火,站在那里烤火;彼得也同他们站着烤火。19 大祭司就以耶稣的门徒和他的教训盘问他。20 耶稣回答说:「我从来是公开对世人说话;我常在会堂和殿里,就是犹太人聚集的地方教训人;我在暗地里并没有说什么。21 你为什么问我呢?可以问那听见的人,我对他们说的是什么;他们知道我说的是什么。」22 耶稣说了这话,旁边站着的一个差役用手掌打他,说:「你这样回答大祭司吗?」23 耶稣说:「我若说得不对,你可以指证那不对之处;若说得对,你为什么打我呢?」24 亚那就把耶稣解到大祭司该亚法那里,仍是捆着解去的。25 西门彼得正站着烤火,有人对他说:「你不也是他的门徒吗?」彼得不承认,说:「我不是。」26 有大祭司的一个仆人,是彼得削掉耳朵那人的亲属,说:「我不是看见你同他在园子里吗?」27 彼得又不承认。立时鸡就叫了。28 众人将耶稣从该亚法那里往衙门内解去,那时是清早。他们自己却不进衙门,恐怕染了污秽,不能吃逾越节的筵席。29 彼拉多就出来,到他们那里,说:「你们告这人是为什么事呢?」30 他们回答说:「这人若不是作恶的,我们就不把他交给你。」31 彼拉多说:「你们自己带他去,按着你们的律法审问他吧!」犹太人说:「我们没有杀人的权柄。」32 这要应验耶稣所说自己将要怎样死的话了。33 彼拉多又进了衙门,叫耶稣来,对他说:「你是犹太人的王吗?」34 耶稣回答说:「这话是你自己说的,还是别人论我对你说的呢?」35 彼拉多说:「我岂是犹太人呢?你本国的人和祭司长把你交给我。你做了什么事呢?」36 耶稣回答说:「我的国不属这世界;我的国若属这世界,我的臣仆必要争战,使我不至于被交给犹太人;只是我的国不属这世界的。」37 彼拉多就对他说:「这样,你是王吗?」耶稣回答说:「你说我是王。我为此而生,也为此来到世间,特为给真理作见证;凡属真理的人就听我的话。」38 彼拉多说:「真理是什么呢?」说了这话,又出来到犹太人那里,对他们说:「我查不出他有什么罪来。39 但你们有个规矩,在逾越节要我给你们释放一个人。你们要我给你们释放犹太人的王吗?」40 他们又喊着说:「不要这人,要巴拉巴!」这巴拉巴是个强盗。

    \textbf{使徒行传 20:28} 圣灵立你们作全群的监督,你们就当为自己谨慎,也为全群谨慎,牧养神的教会,就是他用自己血所买来的。

    \textbf{希伯来书 12:23} 有名录在天上诸长子之会所共聚的总会,和审判众人的神,以及被成全之义人的灵魂,

    \textbf{以弗所书 5:25-27} 25 你们作丈夫的,要爱你们的妻子,正如基督爱教会,为教会舍己,26 要用水藉着道把教会洗净,成为圣洁,27 可以献给自己,作个荣耀的教会,毫无玷污、皱纹等类的病,乃是圣洁没有瑕疵的。

    \textbf{马太福音 24:22} 若不减少那日子,凡有血气的总没有一个得救的;只是为选民,那日子必减少了。

    \textbf{马太福音 24:24} 因为假基督、假先知将要起来,显大神迹、大奇事,倘若能行,连选民也就迷惑了。

    \textbf{马太福音 24:31} 他要差遣使者,用号筒的大声,将他的选民从四方,从天这边到天那边,都招聚了来。

    \textbf{彼得前书 1:1-2} 1 耶稣基督的使徒彼得写信给那分散在本都、加拉太、加帕多家、亚西亚、庇推尼寄居的,2 就是照父神的先见被拣选,藉着圣灵得成圣洁,以致顺服耶稣基督,又蒙他血所洒的人。愿恩惠、平安多多地加给你们!

    \textbf{以弗所书 1:4} 就如神从创立世界以前,在基督里拣选了我们,使我们在他面前成为圣洁,无有瑕疵;

    \textbf{以弗所书 1:7} 我们藉这爱子的血得蒙救赎,过犯得以赦免,乃是照他丰富的恩典。}


我们回到约翰福音十七章和十八章。在此默想的续篇中,并非要处理这些章节的全部内容(那是另一回事),而是从主要目标的光中来看待这一整部分,即神永恒的旨意,及其实陈那旨意的工具或器皿:诸长子的教会。那旨意正如我们所见,乃是在这个宇宙中彰显祂自己,使我们成为祂荣耀的赞美。

因此,就那永恒的旨意而言,我们来到了十字架。在这卷福音书中我们看到,通往神旨意的道路和方法、法则和原则逐步发展,如今十字架凌驾于一切之上。这仿佛一直在指明道路,以及道路上的一切,并在第十七章中展现了道路的终局,即基督在荣耀里。然后圣灵说:「但所有这关于道路和旨意的启示都要求、且唯有藉着十字架方能进入、知晓并实现。」因此在我们这边能够开始之前,我们必须面对十字架的包容性,并看见十字架就神永恒的旨意而言意味着什么。

当然,十字架的完整意义永远无法在短时间内被理解或阐明,但若从两三个角度来审视此问题,可能会对我们有所助益。这样做会使我们看到十字架与神旨意的关系。

\section{从撒旦的角度看十字架}

\subsection{他的仇恨在钉十字架中显露}

首先,我们从钉十字架的角度来看。钉十字架本身与神圣旨意之联系甚为有限。这乃是外来事物的一面,但它代表了非常真实的意义。钉十字架本身并非神思想的一部分,也不是神安排的一部分。基督的死是另一回事,但钉十字架并不是神计划作为整个救赎次序之一必要部分的东西。钉十字架首先代表了撒旦的仇恨和恶意。

我们必须划清界限,区分当我们谈论十字架和谈论钉十字架时的意思。当我们谈论十字架时,我们实际上是指基督之死的全面意义,但当我们谈论钉十字架时,我们指的是一个方面,一种方法;而这种方法和方面代表了魔鬼对神儿子的强烈仇恨和恶意。这是撒旦羞辱并贬低神儿子至最深之处的方式。在世人眼中,没有比钉十字架更可鄙、更令人厌恶、更可怕、更羞辱的死亡形式了。因此藉着钉十字架,撒旦在他对神儿子的真正仇恨和恶意精神中被显露出来。在某种意义上,这是撒旦对祂宣称的回应。在祂公开生活和事工的开始,祂面对天上的宣告:「这是我的爱子……」,并在软弱的时刻、在试炼的时刻、在考验的条件下挑战这一点。他极力强调他的挑战:「如果你是神的儿子……」;而在开始时他被儿子击败了,儿子名分胜过了撒旦。当撒旦——所有邪恶、地狱般和恶魔般事物的化身——攻击祂时,在儿子名分的大能下,在至高的考验中,祂胜过了撒旦,儿子名分得以保持完整。未受玷污、未受污染的儿子名分通过了考验。

从那时起,围绕儿子名分一直有这场持续的争战。一切都围绕着这一点激烈进行。你注意到犹太人用来指控祂应该被钉十字架的理由就是:「祂称自己为神的儿子」。他们永远无法克服这一点,这显示了主所说的「你们是出于你们的父魔鬼」是何等真实。撒旦对那儿子名分及其意义之可怕表达,正是因为那最终所意味的。

最好认识到在祂里面儿子名分受到了多么激烈的攻击,并且在我们里面总是受到激烈的攻击。在宇宙中,鲜有如圣徒之成熟、将神的儿女带到儿子名分更受激烈攻击之事。我们在宝座的光中理解这一点,那宝座最终看到撒旦的整个国度被赶出去,儿子们与儿子一同显现。

然而,在这里你在钉十字架中看到了撒旦对儿子的仇恨,以及儿子名分在祂里面所意味着的,那从来不是出于神的;因为钉十字架就是如此。人们一直认为钉十字架是当时世人所知最可耻或最羞辱的死亡形式。这就是为什么犹太人不会这样做。他们拒绝钉祂。他们把祂带到彼拉多面前。他们强迫彼拉多处理此事。彼拉多说:「你们自己把祂钉十字架吧。」不,他们不会;只有外邦狗才会做这样的事。即使他们道德上已经堕落到如此地步,他们也不会亲手做这事。他们会使用他们认为在世人道德估计和价值观标准上远低于他们的东西来钉十字架。

\subsection{他在钉十字架中对人的掌控显露}

那么还有这额外的因素:钉十字架显明了撒旦对人的掌控是何等牢固;首先,他能使犹太人强迫那个问题,然后使外邦人执行它。撒旦对人的掌控力有多大,竟能利用他们来做这事,对神儿子最大的羞辱。这充分说明了人被魔鬼俘虏的程度,人竟然成为撒旦手中的工具,去做历史上最恶劣的事。如果祂只是一个普通人,被钉十字架就是能对祂做的最糟糕、最贬低的事;但祂是神的儿子!你能理解这意味着什么吗?

现在,把这两件事放在一起,把钉十字架看作是撒旦仇恨的表达,以及撒旦对人的掌控,利用他们煽动那背景仇恨,那么你就有了一个奇妙彰显神圣主权的背景。撒旦所能做的最强烈、最苦毒、最可怕的事,以及藉着那些被他掌控和抓住之人的工具性,成为了两件事的契机:一是撒旦的推翻,二是人类的审判。这就是钉十字架。钉十字架本身被转为对撒旦的推翻。



钉十字架中还有一个额外的元素使其成为那样,当然,那就是我们所说的十字架,基督的死;但藉着钉十字架,撒旦被完全推翻了。在他对神儿子最苦毒、最可怕的攻击的地点和时间,他被遇见并被摧毁,而作为他工具的人受到了他们被用来对神儿子表达的同一审判。

我们必须认识到,在这方面,主耶稣的十字架是世界的审判,正如祂刚刚所说的:「现在这世界受审判了:现在这世界的王要被赶出去」(约十二31);并且这与十字架有关:「我若被举起来……」(约十二32)。有何联系?「我们没有杀人的权柄」(约十八31)。他们不会自己钉祂。这就是使徒对耶稣的话应验的评论的联系(约十八32)。如果你看边注参考,它带你回到12:32:「我若从地上被举起来,就要吸引万人来归我。」因此,犹太人自己不会做的那种举起,而是使用外邦人来做,一方面成为了世界的审判,另一方面成为了世界的王被赶出去。这正是神的主权,就在撒旦最苦毒的敌对之核心。

\section{从神圣的角度看十字架}

然后我们转向从神圣的角度来看它。在这一记载中,这些事是清晰可追溯的。其中之一是主耶稣的十字架是神圣命定的。这是完全明显的。它不是起源于人或魔鬼,而是起源于神圣的经纶。不仅它是神圣命定的,而且它在所有细节上都是神圣治理的。这在这一记录中非常清楚地看到。它是以完美的次序进行的,它是从上头定时的,并且完全在神圣的控制之下。人确曾试图操纵这些程序。他们说:「不要在逾越节。」祂从他们手中接管了这事,确保它在逾越节,因为它与逾越节的意义密切相关。正如保罗后来所说:「基督我们的逾越节。」因此他们会安排时间,但祂从他们手中接管了时间的问题。祂以完美的次序控制它,一点一点地应验了预言的经文。这一记录的一个特征是重复出现这样的话:「为要应验经上的话」,完全有序。这件事在神的手中,而不是在人的手中。总是有另一个角度。

神在这里做主要的事。祂在做什么?毕竟,撒旦和人所做的都是次要的,只是相对的。主要的事是神在做什么。简而言之,神在处理一切破坏永恒旨意的事。神的旨意从古时就已确立,不能被挫败。但既然有很多事破坏了那些旨意的实现,神就要处理一切破坏进来之事,将其清除,并继续朝着实现那旨意前进。

处理一切进来之事并开辟道路、确保祂终局的一个包容性因素是恢复一个照着神圣样式的族类(之人)。再次阅读希伯来书一章和二章,以及诗篇八篇的引述:「人算什么,你竟顾念他?世人算什么,你竟眷顾他?」直译是:「……使他管理……你叫他管理你手所造的」。然后主耶稣被带入那个确切的位置。「我们并未见万物都服在他脚下……但我们见耶稣……得了尊贵荣耀为冠冕」(来二8,9)。祂占据了原本要由第一个亚当占据的位置;当然,超越了那个位置,或者说,第一个亚当是祂的预表,因为经文似乎说他就是那要来者的预像。在这里,藉着十字架,那照着神圣样式的首生的族类被恢复、被确立,并且正是藉着那表达了撒旦强烈仇恨和撒旦在人心中可怕工作的十字架,神正在成全一位元帅:「原来那为万物所属、为万物所本的,要领许多的儿子进荣耀里去,使救他们的元帅因受苦难得以完全,本是合宜的」。彼得谈到「基督的苦难和随后的荣耀」,领许多儿子进荣耀里,救他们的元帅因受苦难得以完全。在那黑暗背景为邪恶、宇宙性暴力包围的十字架中,神正在成全一位带领许多儿子进入荣耀的先锋,因受苦难得以完全。

这并不意味着基督在他的本性中有什么不完美,而是完美的成全,是将完美发展到其完全度量,祂起初是完美的,就像孩子是完美的,但祂从孩子到成人被成全了,那些完美藉着苦难被带到最终的发展。祂里面有忍耐,但藉着苦难变成了完美的忍耐。所有的恩典和美德都没有一丝邪恶的痕迹或污点,但经过火的试炼,它们得到了发展,以至于成为许多儿子所需的完全度量。我们在天上有了一位非常伟大的基督。祂有足够的完美给我们所有人。祂藉着苦难被成全、成熟。

这就是十字架从神圣角度的观点,始终与永恒的旨意相关。保持这一点在视野中,并使你所有的线条都汇聚于此。是撒旦在钉十字架中的活动吗?是的,但你的箭指向神的旨意,不是被击败而是被实现。是基督的苦难和死亡吗?是的,但从神圣的角度看,一切都以旨意的实现为终局。

\section{从教会的角度看十字架}

有哪些伟大的词凌驾于教会之上?它们将教会汇集在自己里面。有两个词支配着(我们可以说)两个时代:「选民」和「救赎」。这两个词完全支配着教会在永恒到永恒的历史。拣选带我们回到永恒之前。保罗使用的词是:「……在创世以前,祂在基督里拣选了我们」,这里的「拣选」与「选举」是同一个词。我们在创世以前就在祂里面被拣选了。「照父神的先见被拣选」,彼得谈到教会时说。这就是为什么我们一开始引用了马太福音中关于选民的经文。你注意到,一切都与选民相关地被治理,「若不减少那日子,凡有血气的总没有一个得救的;只是为选民,那日子必减少了。」我们会稍后回到这一点。我们只是指出另一个词是「救赎」,它特别与选民相关。

现在我们回到「拣选」。关于拣选有很多混乱。没有必要如此。我认为关于拣选和所谓的「拣选教义」的很多麻烦来自于人们假设选民是唯一会被拯救的一方,这不是真的。我们不是在谈论时代,不是在谈论这个时代;我们是在谈论所有时代。你不能读启示录并得出结论说选民是唯一得救的群体。再读一遍启示录,带着这种思想。在某个时候(我们不会说什么时候)会有比选民更多的得救者,但选民会在这个时代得救。这个时代的选民代表了对神来说必须有的东西,因为祂不能被剥夺祂完整的意念。列国将行走在城的光中,那城就是教会,选民。因为这一点没有被认识到,而是假设选民代表得救者,其余的人都要灭亡,所以在这件事上就有了如此的混乱。我们不会走向另一个极端,谈论普救论,但我们说的是,就我们所能看到的,神的话语表明在大群体中有一个身体,在大群体中有一个群体,那内层的群体就是选民,除此之外可能还有更多得救者。神为了得到一个特定的目标而进行了广泛的行动。

你注意到撒旦在这方面也是如此,他会进行非常广泛的行动以确保一个特定的目标。他会杀死所有男婴来得到一个儿子。他追求那儿子名分。所有婴儿都会被剑牺牲,只为得到一个。以色列所有的婴孩都会被剑牺牲,只为在魔鬼的意图中得到一个,希律也是如此。

神也进行广泛的行动。祂撒开祂天上治理的宽网(因为天国就是如此),在那网中会发现各种各样的东西,但祂追求其中的特定之物。田里有宝藏;也许不是主将拥有的世界的唯一部分,而是祂在其中追求的东西。撒旦追求选民,神也追求选民。

这种拣选的事实在以色列的例子中以预表的方式被陈明出来。以色列是一个被拣选的民族,但任何读经的人都不会说所有其他民族都被神注定要灭亡。会有从每个民族中聚集的人,甚至在以色列的日子就有对列国的见证,神会向列国显明怜悯;但以色列在列国中占据了选民的位置。甚至有关于列国的许多预言,包括埃及。以色列作为选民并不是唯一认识神和得救的民族或人群,但他们的拣选是为了一个特定的目的,这就是支配整个启示或拣选教义的东西:就是目的。以色列从列国中被拣选,要在列国中彰显神,这就是教会在拣选中的目的。这正是主耶稣所触及的。祂轻轻地触及了这一点。它还需要等待一段时间才能得到更充分的发展,但祂已经触及了它。「他们本是你的,你将他们赐给我。」神在他们来到基督之前就已经确保了他们。

我们的危险,以及人们在这种真理上总是有的危险,是开始组织它,并在与它相关的事上采取错误的态度。当他们看到人们对福音没有回应时,他们就断定他们不是选民,并永远不再理会他们。一旦我们开始采取这样的态度,我们就会挫败神的旨意,我们完全离开了正确的领域。有一件事是显明的,那就是从我们的角度来看,我们绝不能放弃,绝不能一时接受任何灵魂的失落,绝不能将任何灵魂视为无望。一旦我们开始系统化这样的教义,我们就会立刻发现自己充满了矛盾。

关键是确实有这样的选民,而这选民被淹没在整个罪恶的洪流中,被一切破坏神旨意的事物所淹没。一股邪恶的浪潮席卷了整个创造,那浪潮淹没了选民。它在列国中,在撒旦的国度中,但它在那里。

那么,由于这一点,支配选民的另一个词就是「救赎」。这是「所买赎产业的救赎」;是「赎回归神」。这是救赎的最高表达。

恐怕这方面已经被大大忽略了。我们欢喜,而且理当如此,像这样的话:「从一切不义中救赎我们」;「用祂自己的血救赎我们」;但这还不够。是「赎回归神」。这就是保罗在特别和具体地谈到教会时的意思。

以弗所书中所涉及的是教会。我们不是说其他地方没有,但它在那里确实如此,当保罗在以弗所书中祈求主赐给教会「智慧和启示的灵,使你们真知道祂,你们心中的眼睛被照明」;这是为了他们能知道「祂在圣徒中所得的基业有何等丰盛的荣耀」。神在圣徒中有祂的产业。神在选民中有祂的产业。选民与神最深最高意图和旨意相关。它是神思想的中心和核心。神的完整思想与选民紧密相连,一个中心的身体,其中儿子名分被带到完全;因此所买赎的产业是为神的,「赎回归神」。

我们谈到了利未人,你会记得主说:「他们要归我」;「他们要作为摇祭献给耶和华」;「诸长子的教会」;「他们本是你的,你将他们赐给我……凡我的都是你的。」一直都是为神。

为什么在这方面特别强调为神?因为神把祂自己心中的珍宝与那身体、那选民紧密相连,因此,由于选民卷入了整个人类种族的状态中,选民必须在这种救赎的特定意义上被救赎。救赎的意义是什么?这个词是`apolytrosis',意思是松开,带有付出代价的额外思想,松开。当你说「他救赎了我们」,你的意思是「他把我们松开了」;当你说「我们被赎回归神」,你的意思是「我们被松开归向神」。「从黑暗的权势迁移到他爱子的国里」,但那迁移是一种大能的松开。他在他的血中松开了我们。因此,十字架就是教会的大能松开。这是所买赎产业的救赎。

从教会的角度来看,十字架就是这种奇妙的释放,从一切破坏神圣旨意的暴政和捆绑中释放出来。现在你看到神圣旨意藉着十字架得以实现,就教会而言。

让我们总结一下。拣选、被拣选这个词是神的支配性观念。它的意思是,首先,并非所有人都会在这个时代得救。我们不该说谁会得救,谁不会得救,也不该像我们所说的那样接受任何人的失落。但事实仍然是,认为每个人都会在这个时代得救是愚蠢的。这是其中一个错误教义,导致了错误的企业。它是千禧年后教义的基础,认为所有人都会得救,全世界都会得救。神的话语中没有任何教导说在这个时代所有人都会得救,但神正在做的是从列国中取出一个归祂名下的子民。

其次,神的完整思想与这个时代得救的人紧密相连。我们想要更明确地定义这一点。从神的角度来看,从永恒旨意的角度来看,仅仅把人从黑暗带到光明然后留在那里是不够的,也就是说,有所谓的「福音工作」拯救灵魂,然后让他们继续拯救其他人。从神圣的角度来看,神的完整思想不仅仅是得救。

这解释了第三点,为什么信徒从一开始就聚集在聚会中,为什么聚会的秩序是由圣灵建立的。这就是在主的百姓中承认群体原则的必要性的解释,因为得救者的完整思想在视野中。聚会是为了建造,为了信徒的成熟,而聚会的秩序是属灵成熟的一个重要因素:没有独立,没有自由职业的活动,没有个人和个体的支配和兴趣,而是一个秩序。如果我们破坏了神家中的神圣秩序,我们就会立刻阻止圣灵朝向儿子名分的运行。如果我们违反群体原则,我们就会立刻限制我们自己的属灵成长。

如果你想要任何这方面的证明,看看你周围。我们不是采取批评和指责的态度。不要误解。认识到世界上散布着所谓的「福音使命」,在这些使命中,福音不断地向未得救者传讲,尽管其中大多数人是得救的,并且拒绝任何超过他们所谓的「给罪人的简单福音」的东西。不允许有任何建造的工作,如果引入了,就会有不安。因此你到处都能找到这样的福音使命,有大大小小的人群,已经存在了几十年,但他们仍然是属灵的婴孩,不懂男女的语言,只会孩童的语言,不能承受干粮。没有聚会的秩序,没有真正的群体生活,因此你看到人们在基督里年岁渐长却从未从婴孩期成长起来,这是一种矛盾和悲剧。

起初并不是这样的;再次,让我们注意新约百分之九十九是为信徒写的,这表明神如何看待圣徒的成熟,将他们带到完全成长的重要性。现在祂藉着十字架将教会松开了,心中有着完整的思想,对我们来说,停留在各各他的基本事物上是不够的,而是要认识到各各他包含了并涉及了在永恒之前神格计划中的一切旨意。十字架是一件巨大的事,一直延伸到终局。当你最终到达宝座时,那是宇宙统治、权柄、能力和荣耀的宝座,你看着最终建立的宝座,你会发现宝座中间有一只羔羊。十字架通向宝座;这是神的思想。十字架指向这个宇宙的治理,神的普世彰显。

主要的问题是,在这个时代,神是否在为一个终局工作,这个终局是祂决定在永恒过去要达到的,即藉着一个器皿彰显祂自己,而那器皿与祂以儿子名分的关系相关。也就是说,成熟的属灵生命,包含其全部意义。

因此儿子名分是支配性的词;如果你想要选民的另一个词,那就是神思想中的「儿子」;如果你想要救赎的完整意义,那就是神思想中的儿子名分。

\chapter{教会复活的地位}

\scripture{\textbf{约翰福音 20} 1 七日的第一日清早,天还黑的时候,抹大拉的马利亚来到坟墓那里,看见石头从坟墓挪开了。2 她就跑来见西门彼得和耶稣所爱的那个门徒,对他们说:「有人把主从坟墓里挪了去,我们不知道放在哪里。」3 彼得和那门徒就出来,往坟墓那里去。4 两个人同跑,那门徒比彼得跑得更快,先到了坟墓,5 低头往里看,就见细麻布还放在那里,只是没有进去。6 西门彼得随后也到了,进坟墓里去,就看见细麻布还放在那里,7 又看见耶稣的裹头巾没有和细麻布放在一处,是另在一处卷着。8 先到坟墓的那门徒也进去,看见就信了。9 因为他们还不明白圣经的意思,就是耶稣必要从死里复活。10 于是两个门徒回自己的住处去了。11 马利亚却站在坟墓外面哭。哭的时候,低头往坟墓里看,12 就见两个天使,穿着白衣,在安放耶稣身体的地方坐着,一个在头,一个在脚。13 天使对她说:「妇人,你为什么哭?」她说:「因为有人把我主挪了去,我不知道放在哪里。」14 说了这话,就转过身来,看见耶稣站在那里,却不知道是耶稣。15 耶稣问她说:「妇人,为什么哭?你找谁呢?」马利亚以为是看园的,就对他说:「先生,若是你把他移了去,请告诉我,你把他放在哪里,我便去取他。」16 耶稣说:「马利亚。」马利亚就转过来,用希伯来话对他说:「拉波尼!」(意思是「夫子」)。17 耶稣说:「不要摸我,因我还没有升上去见我的父;你往我弟兄那里去,告诉他们说:『我要升上去见我的父,也是你们的父;见我的神,也是你们的神。』」18 抹大拉的马利亚就去告诉门徒说:「我已经看见了主。」她又将主对她说的这话告诉他们。19 那日(就是七日的第一日)晚上,门徒所在的地方,因怕犹太人,门都关了。耶稣来,站在当中,对他们说:「愿你们平安!」20 说了这话,就把手和肋旁指给他们看。门徒看见主,就喜乐了。21 耶稣又对他们说:「愿你们平安!父怎样差遣了我,我也照样差遣你们。」22 说了这话,就向他们吹一口气,说:「你们受圣灵!23 你们赦免谁的罪,谁的罪就赦免了;你们留下谁的罪,谁的罪就留下了。」24 那十二个门徒中,有称为低土马的多马;耶稣来的时候,他没有和他们同在。25 那些门徒就对他说:「我们已经看见主了!」多马却说:「我非看见他手上的钉痕,用指头探入那钉痕,又用手探入他的肋旁,我总不信。」26 过了八日,门徒又在屋里,多马也和他们同在。门都关了,耶稣来,站在当中说:「愿你们平安!」27 就对多马说:「伸过你的指头来,摸我的手;伸出你的手来,探入我的肋旁。不要疑惑,总要信!」28 多马说:「我的主!我的神!」29 耶稣对他说:「你因看见了我才信;那没有看见就信的有福了。」30 耶稣在门徒面前另外行了许多神迹,没有记在这书上。31 但记这些事要叫你们信耶稣是基督,是神的儿子,并且叫你们信了他,就可以因他的名得生命。

    \textbf{约翰福音 21} 1 这些事以后,耶稣在提比哩亚海边又向门徒显现。他怎样显现记在下面:2 有西门彼得和称为低土马的多马,并加利利的迦拿人拿但业,还有西庇太的两个儿子,又有两个门徒,都在一处。3 西门彼得对他们说:「我打鱼去。」他们说:「我们也和你同去。」他们就出去,上了船;那一夜并没有打着什么。4 天将亮的时候,耶稣站在岸上,门徒却不知道是耶稣。5 耶稣就对他们说:「小子!你们有吃的没有?」他们回答说:「没有。」6 耶稣说:「你们把网撒在船的右边,就必得着。」他们就撒下网去,竟拉不上来了,因为鱼甚多。7 耶稣所爱的那门徒对彼得说:「是主!」那时西门彼得赤着身子,一听见是主,就束上一件外衣,跳在海里。8 其余的门徒离岸不远,约有二百肘,就在小船上把那网鱼拉过来。9 他们上了岸,就看见那里有炭火,上面有鱼,又有饼。10 耶稣对他们说:「把刚才打的鱼拿几条来。」11 西门彼得就去,把网拉到岸上。那网满了大鱼,共一百五十三条;鱼虽这样多,网却没有破。12 耶稣说:「你们来吃早饭。」门徒中没有一个敢问他:「你是谁?」因为知道是主。13 耶稣就来拿饼和鱼给他们。14 耶稣从死里复活以后,向门徒显现,这是第三次。15 他们吃完了早饭,耶稣对西门彼得说:「约翰的儿子西门,你爱我比这些更深吗?」彼得说:「主啊,是的,你知道我爱你。」耶稣对他说:「你喂养我的小羊。」16 耶稣第二次又对他说:「约翰的儿子西门,你爱我吗?」彼得说:「主啊,是的,你知道我爱你。」耶稣说:「你牧养我的羊。」17 第三次对他说:「约翰的儿子西门,你爱我吗?」彼得因为耶稣第三次对他说「你爱我吗」,就忧愁,对耶稣说:「主啊,你是无所不知的;你知道我爱你。」耶稣说:「你喂养我的羊。18 我实实在在地告诉你,你年少的时候,自己束上带子,随意往来;但年老的时候,你要伸出手来,别人要把你束上,带你到不愿意去的地方。」19 耶稣说这话是指着彼得要怎样死,荣耀神。说了这话,就对他说:「你跟从我吧!」20 彼得转过来,看见耶稣所爱的那门徒跟着,就是在晚饭的时候,靠着耶稣胸膛说:「主啊,卖你的是谁?」的那门徒。21 彼得看见他,就问耶稣说:「主啊,这人将来如何?」22 耶稣对他说:「我若要他等到我来的时候,与你何干?你跟从我吧!」23 于是这话传在弟兄中间,说那门徒不死。其实,耶稣不是说他不死,乃是说:「我若要他等到我来的时候,与你何干?」24 为这些事作见证,并且记载这些事的就是这门徒;我们也知道他的见证是真的。25 耶稣所行的事还有许多,若是一一地都写出来,我想,所写的书就是世界也容不下了。}

我们在这系列默想的最后一篇中,正好站在我们一路以来所遵循的根基上,那就是复活的根基。一切都处在复活的光照中。

一开始,作为对这些章节的一般性概览,我们可以再次按照《约翰福音伴侣》的大纲进行,然后或许做一些具体的评论。

这一部分的福音书可以很好地被汇集在希伯来书结尾处使徒的话中:「但愿赐平安的神,就是那凭永约之血、使群羊的大牧人——我主耶稣从死里复活的神,在各样善事上成全你们,叫你们遵行他的旨意;又藉着耶稣基督在你们心里行他所喜悦的事。愿荣耀归给他,直到永永远远。阿们!」你会很容易地将这全面的话语分解成几个部分,并看到这两章如何被安排在这些部分之下。

\section{大牧人归来}

这里我们有大牧人从死里被带回,就是我们的主耶稣。

我们的大纲遵循这样的进程,这些章节是对教会原则上的清晰具体呈现。

首先,它是对基督复活的专属见证;也就是说,他将自己作为复活之主的启示局限于教会,从未给予世界。鉴于有许多人相信或接受他复活的历史事实,但却不能被视为教会的一部分,这必须被理解为不仅仅是耶稣基督从死里复活的事实。它必须包含圣灵在心中对复活之主的启示。这对构成教会成员是必不可少的。教会是由对复活基督的直接、立即、个人启示构成的,这样的启示被视为教会根基不可或缺的。那位被赐予教会独特启示的使徒,也被赐予了复活之主的独特启示。

这是第一件事,即对复活基督的活知识与教会紧密相连,教会从那知识中获得其存在和特性。

其次,因此,他使教会成为一个复活的群体,然后通过首先升到父那里作为其元首(约翰福音20:17),成为一个属天的子民。很明显,在他复活后的四十天期间的某个特定时刻,他确实在父面前显现过,否则我们就无法理解一个看似矛盾的地方,因为在这里他对马利亚说:「不要摸我,因为我还没有升上去见我的父」,但在另一个地方却说他们抓住了他,或抱着他的脚。没有任何责备的记录,没有任何暗示或表明他推开他们,而是明确地说他们确实抓住了他。后来他说:「伸过你的指头来...摸我,看...」。

我们可以把向马利亚的显现视为他复活后的第一次显现,在那之后和随后的显现之间——他们确实抓住了他,他也说「摸我」——必定发生了在父面前的显现,正如这些话所代表的:「我还没有升上去」。

因此,他通过首先在父面前作为教会的元首显现,使他的教会成为属天的子民。我们稍后会更多地看到这一点。

第三,他基于他十字架之血所成就的平安(第19、20和26节)建立教会。教会正是建立在那平安的根基上,他十字架之血所成就的平安。

接下来,他确立了圣灵将成为这个世代教会的治理现实这一事实(第22节):「说了这话,就向他们吹一口气,说:你们受圣灵!」那是前瞻性的,而不是当时实际发生的;也就是说,他们在他向他们吹气的那一刻并没有领受圣灵。这是非常清楚的;但这是一个象征性的行动,确保了他们后来会领受圣灵。象征主义稍后会再次提到,但这里的重点是,那时确立了圣灵将成为这个世代教会的治理现实这一事实。

然后,他又明确指出,与他作为复活者的完全交通的祝福是通过信心(第24和29节)。多马因不信缺席了八天,最终当他出现并被说服时,主说:「那没有看见就信的有福了。」这是一种更大的祝福,与他作为复活者的完全交通的祝福是通过信心。

第六,他赋予教会家庭的美好特征:「你往我弟兄那里去,告诉他们说:我要升上去见我的父,也是你们的父。」

现在,这是对显示教会从主的角度来看是什么的要点和原则的总结。

\section{在复活基础上与他的连接}

进入约翰福音21章,我们注意到这一章是后来的启示。很明显,约翰在第20章第31节结束了他叙述,然后,似乎是新的启示,他添加了第21章的内容。这一章讲述了基督复活后第三次向他们显现的事件。约翰说:「这是第三次耶稣向他们显现。」

我们在这里代表了什么?总的来说,这是在他的子民与他自己之间建立一种新的连接,基于复活所代表的基础。复活代表了一个全新的位置,在那个新位置中,他寻求将他们与他自己连接起来。旧的连接方式已经被打破;那一切都结束了,已经从他们身上被拿走了;仿佛他们悬在天地之间,脚下没有任何坚实的根基。他们的关系是非常不确定和模糊的,在这第三次显现中,他寻求在新的根基上明确新的关系。

这里有些事情我们可以视为象征性的。教会以他们为代表在海上,我们知道海在圣经预表中是人类的象征。仿佛教会在这里被表现为在世界中在人群之中。

然后,他们整夜劳苦却一无所获,原因就是自我努力。彼得说:「我打鱼去。」他们说:「我们也和你同去。」他们整夜劳苦却一无所获。这是在世界中的自我导向、自我驱动的活动,以失败告终。

然而,基督在远处的岸上,知道他们的一切,也知道他们的失败。但当他们最终完全彻底地在他的治理之下时,失败的地方变成了丰盛的地方。在他们的情况下,顺服他的治理意味着放下他们所有的自然推理,「我们整夜劳苦」(捕鱼的最佳时间),在光中撒网对渔夫来说并不明智。如果你在黑暗中失败了,你不太可能在光中成功;但所有这样的自然推理,以及支配天然人在其活动中的一切法则,都必须被放下,或者当我们完全顺服基督的治理时,我们必须愿意放弃它们。这是一回事,即在心思、内心和意志上顺服他的元首权柄。当如此行时,那导致失败的事物,在自然推理会认为完全违背任何希望或期望的情况下,却可能成为丰盛的地方。

因此,教会的丰盛不依赖于有利的环境,而依赖于顺服基督。这是一个原则,一条法则。最不利的环境如果是在顺服主的情况下,可能会非常有果效,而不在主的旨意中,即使自然条件最有利,也可能完全无果效。

然后你注意到鱼的数量的精确性,「一百五十三条」。如果圣灵指示任何记录的写作,我们可以认为他不仅仅是为了形成叙述而使用词语,而是词语对他有分量,如果他启发了那句话的写作,他显然有某种意思。当他们把鱼带到岸上时,他们数了鱼。

我们不会停留在数字的象征意义上。重点是精确性。对我来说,这说的是在这个时代,在基督的指引下,从人类的海洋中聚集的选民,这代表了与他自己的特殊关系。这在第15到18节中被陈明。

我们要暂时停在这最后一段。让我们再次谈谈选民和拣选的问题,我们所说的这一切只是指神的话语陈述有选民这样的事,并揭示选民是与神的旨意相关的选民。支配神拣选的正是特定旨意的实现。换句话说,选民是神的一个观念和思想,目的是为了对主有特别和独特的服事,与他有最亲密的关系,并在他里面最充分地表达他,为他人的好处。

拣选不仅始于也终于救恩的问题。我不知道我们在救恩方面能把它推到多远。当我们谈论选民和拣选时,我们必须始终牢记呼召,因为这是支配它的。选民只是在神的计划中以特殊关系与他自己联系,为了特殊目的。仅此而已(但这是一个伟大的「仅此而已」),这就是我们所说的拣选;一个器皿,一个在神预知中确保的工具。

永远记住:「照父神的先见被拣选」。神知道。神不住在时间中。所有时间在他都是现在;所有我们称为未来的事在神现在的这一刻都存在。他是超越时间的。走出我们人类感官的领域,时间就不再是任何考虑或因素。你知道当你睡觉时你会失去意识。你可能因为某种原因彻底失去意识半小时,当你恢复意识时,可能感觉像过了几年,一生的时间。关键是:走出人类感官的领域,你就走出了时间,神不受人类感官的支配;他超出了我们人类生命的所有这些事物。他将他的旨意提交给运作的年代和年代,在他都是现在的时刻,他知道终点;如果他知道在某个时候,因为某些事情,在他的恩典、他的活动和他的主权工作中,某些人会对呼召作出回应,那么在那种知识中,他可以命定那些人成为他的器皿,为了那个特殊目的。

这使它超出了神选择一些人得救,他们不会失落,而那些没有被选择得救的人可能会得救也可能不会,但在他们的情况下没有任何依据的层面。让我们走出那个领域的事物,看到拣选主要在神的预知中与旨意相关,选民是一个具体的群体。那选民属于这个时代,是在这个时代被聚集的教会。

在我们上一次的默想中,我们说教会、选民不是唯一得救的群体;其他人也会得救。但你记得我们小心地说,我们不关心其他人何时得救的时间因素,我们完全拒绝普救论的思想,即每个被造的存在都会得救,包括魔鬼。我们不能容忍这样的想法。我们不相信这是符合圣经的,无论人们多么巧妙地制定了一个系统,似乎令他们满意地证明了相反的观点。

关键是选民正在这个时代,或这个时代的这部分被聚集,会有一个教会将被提的时刻。但这并不是恩典时代的结束;这不是人类得救的结束。以色列还要进来,以色列不属于教会。教会是完全不同的东西。以色列将得救,列国中也会有其他人得救。这就是我们说选民是为了特定目的的特定群体的意思,构成了这个时代的主要目标。

如果你愿意认为教会的被提之后时代还在继续,我们并不争辩时代可能不会随着教会的被提而结束。在整个人时代结束之前可能还有其他事情发生。但这不是重点,这就是我们所说的拣选和救恩临到教会之外的其他人。

我再次仔细阅读了启示录中的那些书信,从头到尾,有一件事以更新的力量临到我:当你穿过那本书时,你会发现不同的群体在不同的时间在天上,你会发现在一个特定的时刻,当某些群体已经被描述为在天上时,天使带着永远的福音出去。永远的福音中有某种东西,可能与这个时代恩典福音不同,可能与国度的福音不同。我们不声称理解它,但我们看到它在那里,关键是你会在不同的时间在天上发现不同的群体,你会发现范围在扩大,当你来到终点时,你实际上有了羔羊和新妇,羔羊的妻子,你有了被邀请参加婚宴的宾客,然后你有了对更大群体的呼吁(注意,很多人在这里走偏了,我们的诗歌也误导了我们):「圣灵和新妇说:来!听见的人也该说:来!口渴的人也当来;愿意的,都可以白白取生命的水喝。」这一直被认为是圣灵和教会向主耶稣说:「来!」但看看背景。你有生命水的河(启示录22:1),你有教会,新妇,羔羊的妻子。婚宴已经举行,有城,有河,有羔羊和他的新妇已经在那里,然后这:「圣灵和新妇说:来...」(第17节)。对谁?「愿意的,都可以白白取生命的水喝。」这不是呼唤主耶稣来。这是邀请其他人领受那生命,享受那生命,那生命在这里的教会、在城中:「我又看见...新耶路撒冷由神那里从天而降...预备好了,就如新妇妆饰整齐,等候丈夫」(启示录21:2)。你看,如果你倾向于争论教会是永远唯一的得救群体,你有很多要克服的。绝不是这样。

现在,当基督向门徒吹气时,这是一个象征性的行动,暗示或陈明了复活中的新创造。神向第一个亚当吹入生命的气息,他就成了有灵的活人。这就是第一个创造,第一个种族如何成为有生气的,旨在彰显神的荣耀。它失败了,神在基督耶稣里有了一个新创造,从他的死中出来,在他的复活中被 raised up,在那些代表那新创造开端的人身上,他以象征性的行动吹气。他们是神的灵所造的第一个新创造,那创造注定要在基督里向惊奇的宇宙揭示神是什么样子。这就是选民。

因此使徒劝勉:「应当更加殷勤,使你们所蒙的恩召和拣选坚定不移。」使徒从他在希腊世界熟悉的竞赛中得到了这一点,奥林匹克运动会。当赛跑者进入比赛时,在赛道的某个点有一个弯道,从那一点开始是回家的最后一圈,当他们绕过那个弯道时,目标就在眼前,奖赏在望,就在那一点立着一个鼓励的标志,上面的字在英文中意思是「加快速度」。当他们绕过那个弯道,看到那个标志时,聚集在那里的群众会欢呼鼓励,尽一切所能使那些话成为有生命力的话,带着能力:「加快速度!目标在望,现在不要退出,这是最后一圈!」

使徒拿起这一点,说:「应当更加殷勤,使你们所蒙的恩召和拣选坚定不移。你们现在在最后一圈了;不要在这里退出,不要在这里放弃——加快速度!」这难道不是保罗的精神吗?「弟兄们,我不是以为自己已经得着了;我只有一件事,就是忘记背后,努力面前的,向着标竿直跑,要得神在基督耶稣里从上面召我来得的奖赏...我为他已经丢弃万事,看作粪土,为要得着基督。」我加快速度,我更加殷勤。

你看到奖赏是什么,目标是什么。希伯来书告诉我们:「所以,同蒙天召的圣洁弟兄啊...我们见耶稣得了尊贵荣耀为冠冕...要领许多的儿子进荣耀里去。」这是宝座,是选民共享的荣耀,是那些达到儿子名分和成熟、与主同行到底的人的荣耀。这着实是今天给主百姓的话:加快速度!更加殷勤!

响彻四处并将持续响彻的音符是儿子名分、产业、宝座、在将来的世代中与他在治理上的伙伴关系。

愿主将他自己的紧迫感放入我们心中,赐给我们清晰的理解,明白他正试图对我们说什么,并通过我们对他所有百姓说什么。


\chapter{约翰福音伴侣}

\begin{center}
    \textit{原于1934年由「见证与见证」出版社以小册子形式出版。}
\end{center}

\begin{outline}

    \outlinesection{引言}

    约翰福音本身即足以自我说明,但就其更深层含义而言,若干建议或有助益。

    1. 约翰在其所有著作中(福音书、书信和启示录)的主题皆是「耶稣的见证」。

    此见证显明即是基督自己。

    此见证不仅藉着教导,更是藉着与基督生命的联结而延续。


    2. 约翰用于指代「神迹」的特用词为「记号」。意即一切皆为教导,而非仅供趣味或令人惊奇。

    这实际上是「约翰」的关键。一切都具有隐藏的含义。

    所说、所行皆为他事之记号。我们必须更深入地寻找其所指。


    3. 《约翰福音》非属地之历史,乃属灵之历史;关乎天而非仅关乎地;关乎永恒而非仅关乎时间。
    《约翰福音》与《以弗所书》处于同一领域。


    4. 约翰福音乃伟大真理及其法则之全面体现。每一伟大真理皆有其自身法则,顺服那法则乃是进入该真理经历之途。

    \outlinesection{第一章 \\ 呈现}

    1. 从永恒:
    (a) 与神同在者,第1-2节
    (b) 万物是藉着祂造的,第3节
    (c) 生命的泉源,第4节

    2. 进入时间:
    (a) 其先锋——见证人,第6-8,15-42节
    (b) 帐幕中的荣耀(不像摩西的定罪;乃是恩典和真理)第14-17节
    (c) 未被认识之访客,仅为少数人接待,第10-13节
    (d) 神已为自己预备了一只羔羊作祭物,第29,36节

    3. 一群被召聚归祂的人,第43-47节
    第一章包含了整本约翰福音的雏形。(见附录)。

    \outlinesection{第二章}
    「神迹的开始」(第11节)。
    在迦拿婚宴的记号中,一切后续之记号与真理皆以雏形存在。此乃整个「福音」之基础。

    1. 「第三日。」第1节。
    在圣经中,三是神圣见证完全的记号。

    取第一章的内容:
    (a) 基督位格的真理,第11节
    (b) 施洗约翰的见证,第11节
    (c) 门徒的聚集,第11节

    2. 「婚姻。」
    基督与祂教会联合的预表。启示录19:7(同一位作者)。参见以弗所书5:25。
    (a) 水缸——器皿。人类的预表。
    (b) 从空虚到丰盛。基督的群体所享受的。(参见1:16)。
    (c) 从死到生。(「没有酒了。」)基督的群体所享受的。
    (d) 从绝望到喜乐。
    (e) 从羞耻到荣耀。基督的群体所享受的,第11节(参见1:14)。

    酒乃血与生命之预表。血是盟约(婚姻)的媒介。

    关键词:「我的时候,」第4节;「记号,」第11节;「荣耀,」第11节。

    此处的见证是生命胜过死亡,这在第13-25节中与逾越节相关联。

    逾越节:
    (1) 羔羊被杀
    (2) 血被流出。
    (3) 死亡被摧毁。
    (4) 一群人被确保。
    神的国度。

    \outlinesection{第三章}
    神的国度
    神的国度非仅一领域,乃一状态;非仅外在事物之次序,乃生命之状况;非外加之系统,乃源于神之生命与性情。

    1. 处于这国度中的需要和关切。

    2. 支配这国度的法则。「你们必须重生。」
    三件事:
    (1) 差别,第6节
    (2) 本质,第6节
    (3) 基础,第15-18节

    尼哥底母对应于酒用尽了和新生命的奇迹。

    旷野中的铜蛇,第14节。
    (1) 咒诅。
    (2) 天性之人在咒诅之下。(即使是像尼哥底母这般的宗教领袖)。
    (3) 基督为我们成了咒诅,使我们得救。
    (4) 对钉十字架之基督的信心使人脱离咒诅。

    \outlinesection{第四章}
    永恒生命的真理
    尼哥底母代表了天然人在死亡领域中之光景,以及对新生命的需要;因此他为新生命预备了道路。

    1. 当地背景。
    说明缺乏神的生命。
    (1) 属灵方面;持续的缺乏感。持续的不满足。
    (2) 道德方面;生活与神的心意不和谐。
    (3) 宗教方面:无力的传统。宗教反对而非支持。

    此非似生命之生命,实乃死亡。

    2. 永恒生命的性质。
    此乃神自身之生命;与人之生命不同;其特质赋予它胜过死亡的大能,由「永恒」所暗示。

    3. 永恒生命的法则。
    神之圣灵内住,第14节。第7:38,39节。这与基督本人相关,第14节。
    这使一切在属灵上活过来,第23节。

    此处的见证再次是基督里的生命拯救脱离死亡,并在第46-54节中总结。

    \outlinesection{第五章}
    在神大能中行走
    池边的瘫子。
    关键经文:19, 20, 21, 30节。

    在本章中,基督取了天然人之地位,并表明作为这样的人,祂不能凭自己做任何事。

    那人瘫痪了三十八年。

    这是以色列人在旷野中直到摩西死时的旅程时期:以色列的无能和试炼。

    这人象征着在律法褥子上的无能,无法承担律法。

    基督在律法之后以「恩典和真理」(1:17)来到。

    一种新的内在力量使人能够承担律法。

    那人不能凭自己做任何事。生命的道藉着基督临到他,他就行走了(第26节)。

    律法——像褥子一样——原是为了祝福;但人的软弱使它成为捆绑。

    基督拯救人脱离律法的捆绑。

    这种生命和大能行走的法则是:将一切视为从主而来,而非从我们自己而来。这是基督自己道德和属灵超越生命的法则。

    本章中的安息日说的乃是神的安息。

    它与基督相关,因为祂使神的工臻于完美。

    \outlinesection{第六章}
    生命胜过死亡,作为现今且持续之见证(第50节)。
    1. 逾越节临近(第4节)。生命胜过死亡(出埃及记12章)。

    2. 吗哪(第30-32节)。生命胜过死亡(出埃及记16章)。
    (1) 指明藉着羔羊的血从死亡中得到初始的救恩。
    (2) 指明我们藉着不断领受基督而在生命中得蒙保守。

    3. 这胜过死亡的法则。

    以基督为食。
    第53节。「除非你们吃」(原文指一次性的)与逾越节相连。
    第54节。「吃的人」(原文含持续之意)与吗哪相连。

    我们以基督为食藉着:
    (1) 祷告。
    (2) 神的话语。
    (3) 顺服祂。
    (4) 与信徒的交通。
    (5) 敬拜。

    第3章 - 对新生命的需要。
    第4章 - 新生命。
    第5章 - 新的行走。
    第6章 - 新的胜利。

    \outlinesection{第七章}
    新日之预示
    第6章结束了福音书的「生命」部分。

    可以看出,生命唯在基督里才可能,许多人被冒犯并退去。

    第7章是从生命到光的过渡,并结合了两者。光的部分将以不信和筛选结束,正如生命部分一样。

    住棚节是这里的背景。在此节期,巨大的灯台被点亮,从毕士大池取来的大量水在圣殿中被倒出。

    基督抓住此习俗,将自己置于两者的位置,集光与生命之双重象征于一身。

    一个新日子在此被看见——第八日(第37节;利未记23:36),这将是圣灵的日子,基于被荣耀的基督(第39节)。

    本章包含一个重要的秘密。面对几乎普遍之不信(甚至在祂自己的家人中)、敌意、怀疑、偏见和对祂生命的险境,耶稣保持着平静、稳定、强大的道德超越,并像一个受保护的人一样行动,直到祂的工作完成。

    何故?因祂与神有一个秘密的生命,祂拒绝从中被拉出来。祂不照人的命令行动,不受既定宗教礼仪支配,亦不随人们对祂的期望或政治考量,而是照父的内在见证;等候祂的认可和行动的时机(第8, 9节)。

    \outlinesection{第八章和第九章}
    基督被呈现为光
    至第六章末,主题为生命(1:4)。

    第七章是过渡章节。

    第八章引入了光的主题(1:4)。

    第1-11节是引言;犹太领袖们,即便在律法面前亦是瞎眼的,在作为光的基督面前被定罪,内心显露。

    第八章强调了基督是光的事实;天然人在黑暗中;自由源于对真理的认识和顺服;而基督即真理,是神的完全启示。

    第九章是第八章的一个记号(实物教训)(特参第1, 4, 5, 39, 40, 41节)。

    人「生来瞎眼」,此与神的工作有关。

    救恩非藉相信某些教义,乃藉基督赐予一个新的属灵官能,这在我们里面从未运作过。

    这人的状况是所有天然人状况的例证,即便是宗教的犹太人。

    支配这种新活知识的法则是基督的完全主权。不是传统、人、宗教体系;而是个人并完全降服于基督。这贯穿第八至九章。

    这种降服的后果被看作是巨大的代价;被人赶出去。但基督接纳这样的人并更加满足他们。

    \outlinesection{第十章}
    分别归向基督
    第九章以那些降服于基督之人的遭遇结束;他们被赶出去。

    第十章开始讲基督对这样的人做什么;祂带领他们离开一个秩序进入真正的羊圈。

    祂成为他们的牧人。

    第十章标志着一个重大转变。

    至此,一切皆是个人的;现今则是团体的。所有在个别案例中说明的伟大真理,现今都体现在一个被呼召出来的群体中。

    这运动始于归向基督,终于拥有永恒生命。

    基督在这里被看作是:
    (1) 牧人——带领出去,第2, 3节
    (2) 门——带领进入,第7节
    (3) 好牧人——激发信心,第11节
    (4) 一位牧人——带来合一,第16节

    结果:分裂。第31, 42节

    \outlinesection{第十一章和第十二章}
    祂的身体教会

    注意现在正在发生一个明显的运动。基督正在与祂自己的人亲近;公开事工正在停止,而祂专注于祂的教会,为未来的见证(11:54)。

    「伯大尼」(第1节)。

    三幅图景:
    (1) 路加福音10:38。紧张与不和。
    (2) 约翰福音11章。死亡。
    (3) 约翰福音12章。复活中的筵席。

    这是教会的属灵历史和性质。
    (1) 与逾越节相关。因罪有审判和死亡,为着新生命(11:49-51;12:1)。
    (2) 与基督的得荣耀相关(第4节)。
    (3) 在属灵上与天然人无望的状态相关,需要新生命(第39节)。
    注意基督的许多延迟。
    (4) 目标乃一见证基督的器皿(12:10,11;11:52)。
    (5) 结果,对基督和见证器皿的敌对。

    一粒麦子(12:24):
    (1) 少中得多。
    (2) 失中得益。
    (3) 死中得生。

    这就是「伯大尼」。基督和祂的教会。

    \outlinesection{第十三章}

    神的仆人和服事

    作为第1-11章所有属灵真理的体现。

    被拣选的群体现在来到服事。
    第3章 天上的出生。
    第4章 永恒生命。
    第5章 在胜利中行走。
    第6章 生命胜过死亡。
    第7章 圣灵的丰盛。
    第8-9章 属灵启示。
    第10章 分别的群体。
    第11-12章 教会的性质。
    第13章 服事。

    教会要延续基督的职事。

    (1) 罪和败坏进入,因撒旦在骄傲中拒作仆人,妄求与神同等。
    (2) 罪和败坏得处理,藉基督暂时放下与神同等的地位,取了奴仆的形象。
    (3) 教会必须在这方面有基督的心思。(参见腓立比书2章)。

    第13章教导通往荣耀的道路乃藉谦卑、苦难和羞辱,以拯救脱离罪。

    在第13章第34节,「福音」特别进入了爱的主题。

    \outlinesection{第十四章}
    与基督的天上交通

    基督关于离去及其方式所说的开始像冰冷的手一样触动他们。

    万事皆处于不确定状态。这是一件不可避免的事,也是属灵历史的必要部分。自从第12章以来,他们在属灵上就在复活的根基上,这意味着世界被抛在后面,属天的事物取代了属地的事物。因此基督引入了另一个属灵因素——祂离去后与祂自己属灵交通的奥秘;在天上的与基督联合。

    随着一切属地事物的不确定和消逝,祂引入了一个触及整个情况的词。

    Menō(希腊文)意指——停留、留下、常在、继续、持久、永久。

    第2节。(1) 天上的住处(「住处」),原文Monai。
    第10节 (2) 父住在祂里面。
    第17节 (3) 圣灵将住在他们里面。
    第23节 (4) 神格将在信徒里面建立祂的居所。

    他们的问题是:
    (1) 我们如何到神那里去?「我就是道路,」第6节
    (2) 我们如何知道关于神的真理?「我就是真理,」第6节
    (3) 我们如何知道神的生命?「我就是生命,」第6节

    去、知、活,基督即答案。

    \outlinesection{第十五章}
    藉与基督的天上交通结果子

    以色列古时被称为主的葡萄树。

    基督现在取代了以色列的位置:「我是真葡萄树」(第1节)。

    葡萄树的目标是神的荣耀和满足。

    (参见与第2章的联系:酒、葡萄树、荣耀、婚姻、联合、生命、喜乐、丰盛。)

    结果子的法则是「常在基督里」。

    基督的结果子是因为祂常在父里面。

    他们的(和我们的)是要继续表达祂生命的原则。

    祂常在父里面——不在祂自己里面。

    祂藉着顺服父,并非咨询自己亦非顺服恶者来成全之。

    我们常在基督里并多结果子,是藉着寻求从祂而非从我们自己做一切事。

    爱是结果子的秘诀。

    \outlinesection{第十六章}
    祂离去的益处

    基督说圣灵的到来比祂自己的留下更重要(第7节)。

    此系何故?
    (1) 祂的同在是外在的。
    圣灵会在里面。

    (2) 祂一次只能与一个地方的一些人同在。
    圣灵会与各处的所有人同在。

    (3) 祂最多只能停留几年。
    圣灵会在这世代常住。

    (4) 祂来是为了藉代赎的死完成我们救恩的工作。
    圣灵会在全世界使人确信那工作的需要。

    第16章显示,逼迫将来自宗教世界,但圣灵会与他们同在,成为他们的力量。

    这也教导他们的服事装备将与祂的一样。

    \outlinesection{第十七章}
    祭坛旁的祷告

    基督在此取了大祭司之位。祂已经取代了犹太节期、祭物、圣殿、葡萄树等。

    祂现在要献上全燔祭(祂自己)(第19节)。

    祂的祷告将用祂自己的血封印。

    这祷告包括了这本福音书的三个部分,即:

    生命、光和爱。
    (这些都被提及和处理了)。

    这祷告是:
    1. 愿父在子里面得荣耀,第1节
    2. 愿子在父里面得荣耀,第5节
    3. 愿祂在门徒里面得荣耀,第10节
    4. 愿门徒在祂里面得荣耀,第24节
    5. 愿他们都合而为一,第21节
    6. 愿他们蒙保守脱离恶者,第15节

    这祷告应验了吗?

    是的!《使徒行传》这本书显示了应验。

    神在祂复活中在祂里面得荣耀。

    所有信徒因分享一个生命而合一。

    这祷告的答应尚将在普世彰显。

    第17章采用了「约翰」的大部分伟大词汇:生命、光、爱、真理、相信、知道、荣耀、父、子等。

    \outlinesection{第十八章和第十九章}
    基督君王

    当这审讯结束时,除了祂自己之外,没有任何人可以站立的真实根基。

    看祂如何在仇敌中间掌权:
    1. 祂凭祂的位格掌权。
    对士兵和差役:当祂说「我就是」时,他们就退后(第6节)。
    祂那时吩咐他们做什么(第8节)。

    2. 祂凭祂早已说过的话掌权。
    对祂的羊群(17:12)。
    对祂的被否认(13:38)。
    对祂的被卖(13:2,18,21)。
    对祂的死(12:32,33;马可福音10:33)。

    3. 祂使犹太人失去道德立场。
    他们不得不反复改变方法来编造案件。他们指控祂:
    (a) 作恶(第30节)。
    (b) 叛乱(第33节)(暗示)。
    (c) 宗教上的过失(19:7)。
    (d) 与彼拉多抗衡(19:12)。
    他们坚持礼仪上的洁净,却屈身于道德上的无耻(18:28)。
    祂迫使他们说出关于自己的最羞辱的话(19:15)。

    4. 祂使彼拉多慌乱。
    (a) 证明他有罪,接受报告而不获取证据,18:34,35
    (b) 使他躲在犬儒主义的面纱后面,第38节
    (c) 迫使他宣告无罪,第38节
    (d) 驱使他采取权宜之计,第39节
    (e) 显露他的不一致,19:1
    (f) 使他两次重复他的判决,第4,6节
    (g) 发现一个秘密的恐惧(注意「更加」),第8节
    (h) 使他成为一个傀儡,第11节
    (i) 揭露更多的道德软弱,第12-13节
    (j) 证明他是一个纯粹的世俗机会主义者,第12,16节
    (k) 引出一项承认(纵带讽刺),即普世之主权,第19-22节

    基督的死是:
    (1) 有意舍去祂的生命;非被夺去。
    (2) 普遍揭露人的罪和邪恶。
    (3) 预言祂将在公义中普世掌权。

    似乎邪恶处于至高控制的位置,但对经文应验的引用(例如,19:24,36)表明神在治理万有之上。

    \outlinesection{第二十章}
    大牧人归来

    本章给出了教会原则上的美丽而具体的呈现。

    1. 对基督复活的专属见证。祂将(并且总是将)自己作为复活之主的启示局限于祂自己的人,从不给世界。

    2. 祂使教会成为一个复活的群体,然后藉着首先升到父那里作为其元首(第17节),成为一个属天的子民。

    3. 祂使教会建立在祂十字架之血所成就的平安基础上(第19, 20, 26节;参见希伯来书13:20)。

    4. 祂确立了圣灵将成为这个世代教会的治理实际这一事实(第22节)。

    5. 祂明确指出,与祂作为复活者的完全交通的祝福乃藉着信心(第24-29节)。

    6. 祂赋予教会家庭的美好特征(第17节)。「父」、「弟兄」(参见希伯来书2:11-13,17;3:1)。

    \outlinesection{第二十一章}
    (本章是后来的启示。约翰显然在20:31结束了叙述。)

    本章记述基督复活后第三次向属祂之人显现之事(第14节)。

    由于三是神圣完全的数字,我们在这里寻找完成的因素。

    本章的主要特征是什么?

    这乃是在复活所代表的不同基础上,将属祂之人重新连结于祂自己。
    (1) 教会在海上(「海」是人类的圣经预表)。
    (2) 因自我努力而经历失败(第3节)。
    (3) 基督在远处的岸上,知道他们的一切。
    (4) 当他们完全顺服祂的治理(超越天然推理)时,失败之地变成了丰盛之地(第5, 6节)。
    (5) 「一百五十三条」的精确性,意指在此世代,于基督指引下从人类中聚集之选民。这代表了与祂的特殊关系(第15-18节)。

    \outlinesection{约翰福音 \\
        给学生的附录。 \\
        建议大纲。}

    1. 序言(1:1-19)。

    2. 叙述。

    整本福音书是对两个伟大对立面的阐释:
    1. 不信与信心。
    2. 世界与基督和祂的子民。

    叙述分为两个主要部分:
    1. 基督向世界的呈现(1:19-12:5)。
    2. 基督向门徒的启示(13-21)。

    这两个主要部分各有其阶段:
    1. (a) 向世界的呈现(1:19-4:54)。
    (b) 在世界中被认出(3-4)。
    (c) 被世界敌对(6-12:50)。

    在楼上房间。
    2. (a) 揭示服事神的基本心思,并清除被拣选群体内部的敌对(13;参见腓立比书2章)。
    (b) 揭示因祂的离去而产生的教会属灵地位(14)。

    1. (a) 向世界的呈现分为两部分:

    (1) 对祂的见证(1:19-2:11)。
    (2) 祂的工作(2:13-4:54)。

    见证是三重的:
    (1) 施洗约翰 - 旧约时代(1:19-34)。
    (2) 认出祂的门徒(1:35-51)。
    (3) 「记号」(2:1-11)。

    1. (b) 认出是三重的:
    (1) 尼哥底母 - 犹太法利赛人(2:13-3:36)。
    (2) 撒玛利亚人(4:1-32)。
    (3) 王的官员(4:43-54)。

    1. (c) 敌对贯穿始终,与工作和见证并行;两者都越来越强调,主要在犹太地——特别是耶路撒冷。在耶路撒冷及其附近有最大的见证和记号;最终的敌对爆发也在那里。

    在路上。
    (c) 揭示结果子的秘密和性质(15)。

    (d) 揭示祂在新时代与他们同在的方式。(圣灵)(16)。

    (e) 祭坛前的祷告(17)。
    (1) 为自己(1-5)。
    (2) 为在交通中的人(6-19)。
    (3) 为将要在交通中的人(20-26)。


    (f) 最后的场景。
    敌对占了上风。
    祂对不信是死的。
    祂对信心得胜并活着。
    祂的死是生命的源头。

    祂的苦难是自愿的、预定的,并没有掩盖祂的道德荣耀。

    约翰是最后一位使徒作家,本书内容是最后的启示和强调,这一伟大意义在于:在约翰去世前,一种致命倾向开始显露,即将基督之位格分割;也就是说,将两种本性分成两个人。认为神性只是在受洗时与祂联合,在十字架上就离弃了祂。约翰写作是为了驳斥这一点,并确认耶稣基督——神的儿子——的不可分割性。

    第一章用以下词语总结了整本书:

    生命,第4节
    光,第4节
    父,第14节
    荣耀(和荣耀化),第14节
    真理,第17节
    见证,第7节
    世界,第10节
    相信,第7节
    名,第12节
    也参见:

    「知道。」
    「工作。」
    「记号。」
    「审判」和「审判。」
    「常在。」
    「彰显。」
    「永恒生命。」
    「肉体。」
    「爱」和「去爱。」
    「看哪。」
    「黑暗。」
    给出上述每个词和短语在整个福音书中出现的次数;然后注意在福音书的哪一部分有特别的优势。

\end{outline}

\end{document}
